\documentclass[]{article}
\usepackage{lmodern}
\usepackage{amssymb,amsmath}
\usepackage{ifxetex,ifluatex}
\usepackage{fixltx2e} % provides \textsubscript
\ifnum 0\ifxetex 1\fi\ifluatex 1\fi=0 % if pdftex
  \usepackage[T1]{fontenc}
  \usepackage[utf8]{inputenc}
\else % if luatex or xelatex
  \ifxetex
    \usepackage{mathspec}
  \else
    \usepackage{fontspec}
  \fi
  \defaultfontfeatures{Ligatures=TeX,Scale=MatchLowercase}
\fi
% use upquote if available, for straight quotes in verbatim environments
\IfFileExists{upquote.sty}{\usepackage{upquote}}{}
% use microtype if available
\IfFileExists{microtype.sty}{%
\usepackage{microtype}
\UseMicrotypeSet[protrusion]{basicmath} % disable protrusion for tt fonts
}{}
\usepackage[margin=1in]{geometry}
\usepackage{hyperref}
\hypersetup{unicode=true,
            pdftitle={Reto interpolación de una mano},
            pdfborder={0 0 0},
            breaklinks=true}
\urlstyle{same}  % don't use monospace font for urls
\usepackage{color}
\usepackage{fancyvrb}
\newcommand{\VerbBar}{|}
\newcommand{\VERB}{\Verb[commandchars=\\\{\}]}
\DefineVerbatimEnvironment{Highlighting}{Verbatim}{commandchars=\\\{\}}
% Add ',fontsize=\small' for more characters per line
\usepackage{framed}
\definecolor{shadecolor}{RGB}{248,248,248}
\newenvironment{Shaded}{\begin{snugshade}}{\end{snugshade}}
\newcommand{\KeywordTok}[1]{\textcolor[rgb]{0.13,0.29,0.53}{\textbf{#1}}}
\newcommand{\DataTypeTok}[1]{\textcolor[rgb]{0.13,0.29,0.53}{#1}}
\newcommand{\DecValTok}[1]{\textcolor[rgb]{0.00,0.00,0.81}{#1}}
\newcommand{\BaseNTok}[1]{\textcolor[rgb]{0.00,0.00,0.81}{#1}}
\newcommand{\FloatTok}[1]{\textcolor[rgb]{0.00,0.00,0.81}{#1}}
\newcommand{\ConstantTok}[1]{\textcolor[rgb]{0.00,0.00,0.00}{#1}}
\newcommand{\CharTok}[1]{\textcolor[rgb]{0.31,0.60,0.02}{#1}}
\newcommand{\SpecialCharTok}[1]{\textcolor[rgb]{0.00,0.00,0.00}{#1}}
\newcommand{\StringTok}[1]{\textcolor[rgb]{0.31,0.60,0.02}{#1}}
\newcommand{\VerbatimStringTok}[1]{\textcolor[rgb]{0.31,0.60,0.02}{#1}}
\newcommand{\SpecialStringTok}[1]{\textcolor[rgb]{0.31,0.60,0.02}{#1}}
\newcommand{\ImportTok}[1]{#1}
\newcommand{\CommentTok}[1]{\textcolor[rgb]{0.56,0.35,0.01}{\textit{#1}}}
\newcommand{\DocumentationTok}[1]{\textcolor[rgb]{0.56,0.35,0.01}{\textbf{\textit{#1}}}}
\newcommand{\AnnotationTok}[1]{\textcolor[rgb]{0.56,0.35,0.01}{\textbf{\textit{#1}}}}
\newcommand{\CommentVarTok}[1]{\textcolor[rgb]{0.56,0.35,0.01}{\textbf{\textit{#1}}}}
\newcommand{\OtherTok}[1]{\textcolor[rgb]{0.56,0.35,0.01}{#1}}
\newcommand{\FunctionTok}[1]{\textcolor[rgb]{0.00,0.00,0.00}{#1}}
\newcommand{\VariableTok}[1]{\textcolor[rgb]{0.00,0.00,0.00}{#1}}
\newcommand{\ControlFlowTok}[1]{\textcolor[rgb]{0.13,0.29,0.53}{\textbf{#1}}}
\newcommand{\OperatorTok}[1]{\textcolor[rgb]{0.81,0.36,0.00}{\textbf{#1}}}
\newcommand{\BuiltInTok}[1]{#1}
\newcommand{\ExtensionTok}[1]{#1}
\newcommand{\PreprocessorTok}[1]{\textcolor[rgb]{0.56,0.35,0.01}{\textit{#1}}}
\newcommand{\AttributeTok}[1]{\textcolor[rgb]{0.77,0.63,0.00}{#1}}
\newcommand{\RegionMarkerTok}[1]{#1}
\newcommand{\InformationTok}[1]{\textcolor[rgb]{0.56,0.35,0.01}{\textbf{\textit{#1}}}}
\newcommand{\WarningTok}[1]{\textcolor[rgb]{0.56,0.35,0.01}{\textbf{\textit{#1}}}}
\newcommand{\AlertTok}[1]{\textcolor[rgb]{0.94,0.16,0.16}{#1}}
\newcommand{\ErrorTok}[1]{\textcolor[rgb]{0.64,0.00,0.00}{\textbf{#1}}}
\newcommand{\NormalTok}[1]{#1}
\usepackage{graphicx,grffile}
\makeatletter
\def\maxwidth{\ifdim\Gin@nat@width>\linewidth\linewidth\else\Gin@nat@width\fi}
\def\maxheight{\ifdim\Gin@nat@height>\textheight\textheight\else\Gin@nat@height\fi}
\makeatother
% Scale images if necessary, so that they will not overflow the page
% margins by default, and it is still possible to overwrite the defaults
% using explicit options in \includegraphics[width, height, ...]{}
\setkeys{Gin}{width=\maxwidth,height=\maxheight,keepaspectratio}
\IfFileExists{parskip.sty}{%
\usepackage{parskip}
}{% else
\setlength{\parindent}{0pt}
\setlength{\parskip}{6pt plus 2pt minus 1pt}
}
\setlength{\emergencystretch}{3em}  % prevent overfull lines
\providecommand{\tightlist}{%
  \setlength{\itemsep}{0pt}\setlength{\parskip}{0pt}}
\setcounter{secnumdepth}{0}
% Redefines (sub)paragraphs to behave more like sections
\ifx\paragraph\undefined\else
\let\oldparagraph\paragraph
\renewcommand{\paragraph}[1]{\oldparagraph{#1}\mbox{}}
\fi
\ifx\subparagraph\undefined\else
\let\oldsubparagraph\subparagraph
\renewcommand{\subparagraph}[1]{\oldsubparagraph{#1}\mbox{}}
\fi

%%% Use protect on footnotes to avoid problems with footnotes in titles
\let\rmarkdownfootnote\footnote%
\def\footnote{\protect\rmarkdownfootnote}

%%% Change title format to be more compact
\usepackage{titling}

% Create subtitle command for use in maketitle
\newcommand{\subtitle}[1]{
  \posttitle{
    \begin{center}\large#1\end{center}
    }
}

\setlength{\droptitle}{-2em}

  \title{Reto interpolación de una mano}
    \pretitle{\vspace{\droptitle}\centering\huge}
  \posttitle{\par}
    \author{}
    \preauthor{}\postauthor{}
    \date{}
    \predate{}\postdate{}
  

\begin{document}
\maketitle

\subsection{Analisis Numerico 2019-1}\label{analisis-numerico-2019-1}

\subsubsection{Por Rafael Salvador Frieri Cabrera, Daniel Hamilton-Smith
Santa Cruz,Laura Juliana Mora
Páez}\label{por-rafael-salvador-frieri-cabrera-daniel-hamilton-smith-santa-cruzlaura-juliana-mora-paez}

\subsection{Introducción:}\label{introduccion}

Para afianzar los conocimientos adquiridos acerca de interpolación en la
clase de Análisis Numérico se propuso una actividad a modo de reto para
los estudiantes. Donde el reto consiste en el siguiente: ``Construir un
Interpolar (no necesariamente en forma polinómica) utilizando la menor
cantidad de puntos puntos k y reproducir el dibujo completo de la mano
(mejor exactitud) con la información dada en el script.''

El script contaba con las coordenadas (x,y) de 67 puntos, los cuales al
unirse forman la silueta de una mano; estas coordenadas vienen en 2
vectores (x,y) separadas por comas de la siguiente manera:

\begin{Shaded}
\begin{Highlighting}[]
\NormalTok{x=}\KeywordTok{c}\NormalTok{(}\FloatTok{14.65}\NormalTok{, }\FloatTok{14.71}\NormalTok{, }\FloatTok{14.6}\NormalTok{, }\FloatTok{14.8}\NormalTok{, }\FloatTok{15.2}\NormalTok{, }\FloatTok{15.6}\NormalTok{, }\FloatTok{15.7}\NormalTok{, }\FloatTok{17.0}\NormalTok{, }\FloatTok{17.6}\NormalTok{, }\FloatTok{17.52}\NormalTok{, }\FloatTok{17.3}\NormalTok{, }\FloatTok{16.8}\NormalTok{, }\FloatTok{15.4}\NormalTok{, }\FloatTok{14.83}\NormalTok{, }\FloatTok{14.4}\NormalTok{, }\FloatTok{14.5}\NormalTok{, }\FloatTok{15.0}\NormalTok{, }\FloatTok{15.1}\NormalTok{, }\FloatTok{15.0}\NormalTok{, }\FloatTok{14.9}\NormalTok{, }\FloatTok{14.6}\NormalTok{, }\FloatTok{14.3}\NormalTok{, }\FloatTok{14.0}\NormalTok{, }\FloatTok{13.9}\NormalTok{, }\FloatTok{13.8}\NormalTok{, }\FloatTok{13.5}\NormalTok{, }\FloatTok{13.1}\NormalTok{, }\FloatTok{13.0}\NormalTok{, }\FloatTok{13.3}\NormalTok{, }\FloatTok{13.2}\NormalTok{, }\FloatTok{13.1}\NormalTok{, }\FloatTok{12.9}\NormalTok{, }\FloatTok{12.4}\NormalTok{, }\FloatTok{11.9}\NormalTok{, }\FloatTok{11.7}\NormalTok{, }\FloatTok{11.6}\NormalTok{, }\FloatTok{11.3}\NormalTok{, }\FloatTok{10.9}\NormalTok{, }\FloatTok{10.7}\NormalTok{, }\FloatTok{10.6}\NormalTok{, }\FloatTok{10.6}\NormalTok{, }\FloatTok{10.1}\NormalTok{, }\FloatTok{9.7}\NormalTok{, }\FloatTok{9.4}\NormalTok{, }\FloatTok{9.3}\NormalTok{, }\FloatTok{9.6}\NormalTok{, }\FloatTok{9.9}\NormalTok{, }\FloatTok{10.1}\NormalTok{, }\FloatTok{10.2}\NormalTok{, }\FloatTok{10.3}\NormalTok{, }\FloatTok{9.10}\NormalTok{, }\FloatTok{8.6}\NormalTok{, }\FloatTok{7.5}\NormalTok{, }\FloatTok{7.0}\NormalTok{, }\FloatTok{6.7}\NormalTok{, }\FloatTok{6.6}\NormalTok{, }\FloatTok{7.70}\NormalTok{, }\FloatTok{8.00}\NormalTok{, }\FloatTok{8.10}\NormalTok{, }\FloatTok{8.40}\NormalTok{,}\FloatTok{9.20}\NormalTok{, }\FloatTok{9.30}\NormalTok{, }\DecValTok{10}\NormalTok{, }\FloatTok{10.2}\NormalTok{, }\FloatTok{10.3}\NormalTok{, }\FloatTok{10.0}\NormalTok{, }\FloatTok{9.50}\NormalTok{)                                                                                                       }
\NormalTok{y=}\KeywordTok{c}\NormalTok{(}\FloatTok{14.7}\NormalTok{, }\FloatTok{14.33}\NormalTok{, }\FloatTok{13.4}\NormalTok{, }\FloatTok{12.33}\NormalTok{, }\FloatTok{11.0}\NormalTok{, }\FloatTok{10.5}\NormalTok{, }\FloatTok{10.22}\NormalTok{, }\FloatTok{8.20}\NormalTok{, }\FloatTok{7.10}\NormalTok{, }\FloatTok{6.70}\NormalTok{, }\FloatTok{6.60}\NormalTok{, }\FloatTok{6.80}\NormalTok{, }\FloatTok{8.30}\NormalTok{, }\FloatTok{8.80}\NormalTok{, }\FloatTok{9.30}\NormalTok{, }\FloatTok{8.80}\NormalTok{, }\FloatTok{6.30}\NormalTok{, }\FloatTok{5.50}\NormalTok{, }\FloatTok{5.00}\NormalTok{, }\FloatTok{4.70}\NormalTok{, }\FloatTok{4.60}\NormalTok{, }\FloatTok{4.50}\NormalTok{, }\FloatTok{4.90}\NormalTok{, }\FloatTok{5.40}\NormalTok{, }\FloatTok{5.80}\NormalTok{, }\FloatTok{6.90}\NormalTok{, }\FloatTok{8.20}\NormalTok{, }\FloatTok{7.60}\NormalTok{, }\FloatTok{5.80}\NormalTok{, }\FloatTok{4.50}\NormalTok{, }\FloatTok{4.30}\NormalTok{, }\FloatTok{3.90}\NormalTok{, }\FloatTok{4.20}\NormalTok{, }\FloatTok{5.70}\NormalTok{, }\FloatTok{7.00}\NormalTok{, }\FloatTok{7.90}\NormalTok{, }\FloatTok{8.20}\NormalTok{, }\FloatTok{7.30}\NormalTok{, }\FloatTok{6.70}\NormalTok{, }\FloatTok{5.50}\NormalTok{, }\FloatTok{5.10}\NormalTok{, }\FloatTok{4.60}\NormalTok{, }\FloatTok{4.7}\NormalTok{, }\FloatTok{5.0}\NormalTok{, }\FloatTok{5.5}\NormalTok{, }\FloatTok{7.2}\NormalTok{, }\FloatTok{7.8}\NormalTok{, }\FloatTok{8.60}\NormalTok{, }\FloatTok{9.40}\NormalTok{, }\FloatTok{10.0}\NormalTok{, }\FloatTok{10.7}\NormalTok{, }\FloatTok{9.9}\NormalTok{, }\FloatTok{9.0}\NormalTok{, }\FloatTok{9.1}\NormalTok{, }\FloatTok{9.3}\NormalTok{, }\FloatTok{9.7}\NormalTok{, }\FloatTok{11.7}\NormalTok{, }\FloatTok{12.3}\NormalTok{, }\FloatTok{12.5}\NormalTok{, }\FloatTok{13.0}\NormalTok{,}\FloatTok{13.91}\NormalTok{, }\FloatTok{14.9}\NormalTok{, }\DecValTok{16}\NormalTok{, }\FloatTok{16.4}\NormalTok{, }\FloatTok{16.8}\NormalTok{, }\FloatTok{10.7}\NormalTok{, }\FloatTok{11.0}\NormalTok{)}
\end{Highlighting}
\end{Shaded}

Tras estos vectores en el script se encuentran dos instrucciones donde
una nos menciona la cantidad de puntos y la otra sirve para graficar en
R:

\begin{Shaded}
\begin{Highlighting}[]
\KeywordTok{length}\NormalTok{(x)}
\end{Highlighting}
\end{Shaded}

\begin{verbatim}
## [1] 67
\end{verbatim}

\begin{Shaded}
\begin{Highlighting}[]
\KeywordTok{plot}\NormalTok{(x,y, }\DataTypeTok{pch=}\DecValTok{7}\NormalTok{, }\DataTypeTok{cex=}\FloatTok{0.5}\NormalTok{, }\DataTypeTok{col =} \StringTok{"red"}\NormalTok{, }\DataTypeTok{asp=}\DecValTok{1}\NormalTok{,}\DataTypeTok{xlab=}\StringTok{"X"}\NormalTok{, }\DataTypeTok{ylab=}\StringTok{"Y"}\NormalTok{, }\DataTypeTok{main=}\StringTok{"Diagrama "}\NormalTok{)}
\end{Highlighting}
\end{Shaded}

\includegraphics{Documento_notebook_files/figure-latex/unnamed-chunk-2-1.pdf}

Tras ejecutar el script se obtiene la gráfica anterior.

\subsection{Criterios}\label{criterios}

\begin{enumerate}
\def\labelenumi{\arabic{enumi}.}
\tightlist
\item
  Metodología que explique como se seleccionaron k puntos con
  k\textless{}n con n el total de puntos dados(Selección ``inteligente
  de los puntos'')
\item
  Algoritmo que se aplico(justificación) aplico por
  ejemplo,interpolación polinomica y como soluciono el sistema
\item
  Validación del resultado
\end{enumerate}

\subsection{Productos}\label{productos}

\begin{enumerate}
\def\labelenumi{\arabic{enumi}.}
\tightlist
\item
  Algoritmo,requerimientos,codificación
\item
  1Codificación, tabla donde esta la interpolación en los n-k puntos (no
  seleccionados), el polinomio o la función interpolante. En un plano
  los puntos originales, los utilizados y el contorno de la mano
  original y el interpolado(utilice el grosor minimo para la curva) 2.2
  Calcular la cota de error de su método con los datos experimentales y
  comparela con la cota teorica 3.1 tabla donde esten los valores
  interpolados(tenga en cuenta los que no utilizo), los originales y el
  error relativo , calcule un error relativo total como la suma de los
  errores relativos 3.2 cree una funcion que cuente el numero aciertos y
  el numero de diferencias en una cifra entre su función de
  interpolacion y los originales y implementelo como el índice de
  Jaccard
\item
  Cree una función que muestre la eficiencia de su metodo
\end{enumerate}

\subsection{Preguntas}\label{preguntas}

¿El origen se puede modificar? Si tenemos nueva información (nodos)
¿como podemos implementar esa información en el algoritmo de
interpolación? ¿Su método es robusto, en el sentido que si se tienen más
puntos la exactitud no disminuye? Si la información adicional o suponga
tiene la información de otra mano con más cifras significativas ¿como se
comporta su algoritmo ? ¿la exactitud decae?

\subsection{Solución del Reto}\label{solucion-del-reto}

El algoritmo implementado tiene como objetivos encontrar K puntos para
formar n cantidad de segmentos donde el error sea mínimo entre la
función interpolada y los puntos de vector original, minimizando la
cantidad de puntos utilizados, al igual que graficar estos segmentos
para obtener la silueta de una mano.

Para cumplir con el objetivo se contemplaron varias opciones, como por
ejemplo sacar el promedio entre cada par de puntos y así reducir la
cantidad de puntos utilizados hasta la mitad, sin embargo esta
posibilidad fue descartada debido al error apreciable que se presentaba
en algunos segmentos de la silueta, por otro lado se consideró la
posibilidad de dividir la mano en fragmentos para luego unirla. Otra
opción contemplada se basó en invertir los puntos de la mano para
obtener la interpolación de la mano rotada y luego devolverla a su
estado inicial, sin embargo se descartó por la complejidad que
presentaba este algoritmo, sin tener un efecto apreciable en la
optimización del uso de puntos. Sin embargo para todas las versiones se
requeria que se ordenaran los puntos de la siguiente manera:

Puntos ordenados

\begin{Shaded}
\begin{Highlighting}[]
\NormalTok{x =}\StringTok{ }\KeywordTok{c}\NormalTok{(}\FloatTok{14.65}\NormalTok{, }\FloatTok{14.71}\NormalTok{, }\FloatTok{14.6}\NormalTok{, }\FloatTok{14.8}\NormalTok{, }\FloatTok{15.2}\NormalTok{, }\FloatTok{15.6}\NormalTok{, }\FloatTok{15.7}\NormalTok{, }\FloatTok{17.0}\NormalTok{, }\FloatTok{17.6}\NormalTok{, }\FloatTok{17.52}\NormalTok{, }\FloatTok{17.3}\NormalTok{, }
  \FloatTok{16.8}\NormalTok{, }\FloatTok{15.4}\NormalTok{, }\FloatTok{14.83}\NormalTok{, }\FloatTok{14.4}\NormalTok{, }\FloatTok{14.5}\NormalTok{, }
  \FloatTok{15.0}\NormalTok{, }\FloatTok{15.1}\NormalTok{, }\FloatTok{15.0}\NormalTok{, }\FloatTok{14.9}\NormalTok{, }\FloatTok{14.6}\NormalTok{, }\FloatTok{14.3}\NormalTok{, }\FloatTok{14.0}\NormalTok{, }\FloatTok{13.9}\NormalTok{, }\FloatTok{13.8}\NormalTok{, }\FloatTok{13.5}\NormalTok{, }\FloatTok{13.1}\NormalTok{, }\FloatTok{13.0}\NormalTok{, }
  \FloatTok{13.3}\NormalTok{, }\FloatTok{13.2}\NormalTok{, }\FloatTok{13.1}\NormalTok{, }\FloatTok{12.9}\NormalTok{, }\FloatTok{12.4}\NormalTok{, }\FloatTok{11.9}\NormalTok{, }\FloatTok{11.7}\NormalTok{, }\FloatTok{11.6}\NormalTok{, }\FloatTok{11.3}\NormalTok{, }\FloatTok{10.9}\NormalTok{, }
  \FloatTok{10.7}\NormalTok{, }\FloatTok{10.6}\NormalTok{, }\FloatTok{10.6}\NormalTok{, }\FloatTok{10.1}\NormalTok{, }\FloatTok{9.7}\NormalTok{, }\FloatTok{9.4}\NormalTok{, }\FloatTok{9.3}\NormalTok{, }\FloatTok{9.6}\NormalTok{, }\FloatTok{9.9}\NormalTok{, }\FloatTok{10.1}\NormalTok{, }\FloatTok{10.2}\NormalTok{, }\FloatTok{10.3}\NormalTok{,  }\FloatTok{10.0}\NormalTok{, }\FloatTok{9.5}\NormalTok{, }
  \FloatTok{8.6}\NormalTok{, }\FloatTok{7.5}\NormalTok{, }\FloatTok{7.0}\NormalTok{, }\FloatTok{6.7}\NormalTok{, }\FloatTok{6.6}\NormalTok{, }\FloatTok{7.7}\NormalTok{, }
  \FloatTok{8.0}\NormalTok{, }\FloatTok{8.1}\NormalTok{, }\FloatTok{8.4}\NormalTok{, }\FloatTok{9.2}\NormalTok{, }\FloatTok{9.3}\NormalTok{, }\DecValTok{10}\NormalTok{, }\FloatTok{10.2}\NormalTok{, }\FloatTok{10.3}\NormalTok{)}

\NormalTok{y =}\StringTok{ }\KeywordTok{c}\NormalTok{(}\FloatTok{14.7}\NormalTok{, }\FloatTok{14.33}\NormalTok{, }\FloatTok{13.4}\NormalTok{, }\FloatTok{12.33}\NormalTok{, }\FloatTok{11.0}\NormalTok{, }\FloatTok{10.5}\NormalTok{, }\FloatTok{10.22}\NormalTok{, }\FloatTok{8.2}\NormalTok{, }\FloatTok{7.1}\NormalTok{, }\FloatTok{6.7}\NormalTok{, }\FloatTok{6.6}\NormalTok{, }
  \FloatTok{6.8}\NormalTok{, }\FloatTok{8.3}\NormalTok{, }\FloatTok{8.8}\NormalTok{, }\FloatTok{9.3}\NormalTok{, }\FloatTok{8.8}\NormalTok{, }
  \FloatTok{6.3}\NormalTok{, }\FloatTok{5.5}\NormalTok{, }\FloatTok{5.0}\NormalTok{, }\FloatTok{4.7}\NormalTok{, }\FloatTok{4.6}\NormalTok{, }\FloatTok{4.5}\NormalTok{, }\FloatTok{4.9}\NormalTok{, }\FloatTok{5.4}\NormalTok{, }\FloatTok{5.8}\NormalTok{, }\FloatTok{6.9}\NormalTok{, }\FloatTok{8.2}\NormalTok{, }\FloatTok{7.6}\NormalTok{, }
  \FloatTok{5.8}\NormalTok{, }\FloatTok{4.5}\NormalTok{, }\FloatTok{4.3}\NormalTok{, }\FloatTok{3.9}\NormalTok{, }\FloatTok{4.2}\NormalTok{, }\FloatTok{5.7}\NormalTok{, }\FloatTok{7.0}\NormalTok{, }\FloatTok{7.9}\NormalTok{, }\FloatTok{8.2}\NormalTok{, }\FloatTok{7.3}\NormalTok{, }
  \FloatTok{6.7}\NormalTok{, }\FloatTok{5.5}\NormalTok{, }\FloatTok{5.1}\NormalTok{, }\FloatTok{4.6}\NormalTok{, }\FloatTok{4.7}\NormalTok{, }\FloatTok{5.0}\NormalTok{, }\FloatTok{5.5}\NormalTok{, }\FloatTok{7.2}\NormalTok{, }\FloatTok{7.8}\NormalTok{, }\FloatTok{8.6}\NormalTok{, }\FloatTok{9.4}\NormalTok{, }\FloatTok{10.0}\NormalTok{,  }\FloatTok{10.7}\NormalTok{, }\FloatTok{11.0}\NormalTok{, }
  \FloatTok{9.9}\NormalTok{, }\FloatTok{9.0}\NormalTok{, }\FloatTok{9.1}\NormalTok{, }\FloatTok{9.3}\NormalTok{, }\FloatTok{9.7}\NormalTok{, }\FloatTok{11.7}\NormalTok{, }
  \FloatTok{12.3}\NormalTok{, }\FloatTok{12.5}\NormalTok{, }\FloatTok{13.0}\NormalTok{, }\FloatTok{13.91}\NormalTok{, }\FloatTok{14.9}\NormalTok{, }\DecValTok{16}\NormalTok{, }\FloatTok{16.4}\NormalTok{, }\FloatTok{16.8}\NormalTok{)}
\end{Highlighting}
\end{Shaded}

Las diferentes versiónes del código se presentaran acontinuación con sus
respectivas soluciones a la mano.

\subsubsection{Versión 1:}\label{version-1}

A continuación se presenta la primera versión que se tuvo del código
donde se utilizaba el método de lagrange, esta versión fue descartada
debido a que a pesar de que solo se utilizaran 15 puntos a la hora de
gráficar las curvas no generaban una mano, si no como se puede observar
quedaban una sobre otra o con una dirección diferente a la esperada, sin
embargo podemos observar que podemos tener algo cercano a el dedo
meñique.\\
Para el calculo de cuales puntos se van a utilizar se tomo la decisión
de dividir en 7 secciones el vector con todos los puntos y de estas
secciones sacar al rededor de 3 puntos. A continuación se presenta el
código con las diferentes salidas que este presentaba y la gráfica de
como quedaba la mano.

\begin{Shaded}
\begin{Highlighting}[]
\KeywordTok{rm}\NormalTok{(}\DataTypeTok{list=}\KeywordTok{ls}\NormalTok{())}

\NormalTok{graf <-}\StringTok{ }\ControlFlowTok{function}\NormalTok{(arr1,arr2, color)}
\NormalTok{\{}
  \KeywordTok{points}\NormalTok{(arr1,arr2, }\DataTypeTok{pch=}\DecValTok{7}\NormalTok{, }\DataTypeTok{cex=}\FloatTok{0.5}\NormalTok{, }\DataTypeTok{col =}\NormalTok{ color, }\DataTypeTok{asp=}\DecValTok{1}\NormalTok{,}\DataTypeTok{xlab=}\StringTok{"X"}\NormalTok{, }\DataTypeTok{ylab=}\StringTok{"Y"}\NormalTok{, }\DataTypeTok{main=}\StringTok{"Diagrama "}\NormalTok{)}
\NormalTok{\}}

\NormalTok{lagrange<-}\StringTok{ }\ControlFlowTok{function}\NormalTok{(eval, lis, lisy )}
\NormalTok{\{}
\NormalTok{  ret1 <-}\StringTok{ }\NormalTok{( eval }\OperatorTok{-}\StringTok{ }\NormalTok{lis[}\DecValTok{2}\NormalTok{] ) }\OperatorTok{/}\StringTok{ }\NormalTok{( lis[}\DecValTok{1}\NormalTok{] }\OperatorTok{-}\StringTok{ }\NormalTok{lis[}\DecValTok{2}\NormalTok{] ) }\OperatorTok{*}\StringTok{ }\NormalTok{( eval }\OperatorTok{-}\StringTok{ }\NormalTok{lis[}\DecValTok{3}\NormalTok{] ) }\OperatorTok{/}\StringTok{ }\NormalTok{( lis[}\DecValTok{1}\NormalTok{] }\OperatorTok{-}\StringTok{ }\NormalTok{lis[}\DecValTok{3}\NormalTok{] ) }\OperatorTok{*}\StringTok{ }\NormalTok{lisy[}\DecValTok{1}\NormalTok{]}
  
\NormalTok{  ret2 <-}\StringTok{ }\NormalTok{( eval }\OperatorTok{-}\StringTok{ }\NormalTok{lis[}\DecValTok{1}\NormalTok{] ) }\OperatorTok{/}\StringTok{ }\NormalTok{( lis[}\DecValTok{2}\NormalTok{] }\OperatorTok{-}\StringTok{ }\NormalTok{lis[}\DecValTok{1}\NormalTok{] ) }\OperatorTok{*}\StringTok{ }\NormalTok{( eval }\OperatorTok{-}\StringTok{ }\NormalTok{lis[}\DecValTok{3}\NormalTok{] ) }\OperatorTok{/}\StringTok{ }\NormalTok{( lis[}\DecValTok{2}\NormalTok{] }\OperatorTok{-}\StringTok{ }\NormalTok{lis[}\DecValTok{3}\NormalTok{] ) }\OperatorTok{*}\StringTok{ }\NormalTok{lisy[}\DecValTok{2}\NormalTok{]}
  
\NormalTok{  ret3 <-}\StringTok{ }\NormalTok{( eval }\OperatorTok{-}\StringTok{ }\NormalTok{lis[}\DecValTok{1}\NormalTok{] ) }\OperatorTok{/}\StringTok{ }\NormalTok{( lis[}\DecValTok{3}\NormalTok{] }\OperatorTok{-}\StringTok{ }\NormalTok{lis[}\DecValTok{1}\NormalTok{] ) }\OperatorTok{*}\StringTok{ }\NormalTok{( eval }\OperatorTok{-}\StringTok{ }\NormalTok{lis[}\DecValTok{2}\NormalTok{] ) }\OperatorTok{/}\StringTok{ }\NormalTok{( lis[}\DecValTok{3}\NormalTok{] }\OperatorTok{-}\StringTok{ }\NormalTok{lis[}\DecValTok{2}\NormalTok{] ) }\OperatorTok{*}\StringTok{ }\NormalTok{lisy[}\DecValTok{3}\NormalTok{]}
  
  \KeywordTok{return}\NormalTok{ (ret1 }\OperatorTok{+}\StringTok{ }\NormalTok{ret2 }\OperatorTok{+}\StringTok{ }\NormalTok{ret3)}
\NormalTok{\}}

\NormalTok{lag<-}\StringTok{ }\ControlFlowTok{function}\NormalTok{(eval, lis, lisy )}
\NormalTok{\{ }
\NormalTok{  tot <-}\StringTok{ }\DecValTok{0}
  \ControlFlowTok{for}\NormalTok{( j }\ControlFlowTok{in} \DecValTok{1}\OperatorTok{:}\NormalTok{(}\KeywordTok{length}\NormalTok{(lis) ) )}
\NormalTok{  \{}
\NormalTok{    val <-}\StringTok{ }\NormalTok{lisy[j]}
    \ControlFlowTok{for}\NormalTok{( i }\ControlFlowTok{in} \DecValTok{1}\OperatorTok{:}\KeywordTok{length}\NormalTok{(lis)  )}
\NormalTok{    \{}
      \ControlFlowTok{if}\NormalTok{ ( i }\OperatorTok{!=}\StringTok{ }\NormalTok{j)}
\NormalTok{      \{}
\NormalTok{        nn <-}\StringTok{ }\NormalTok{( eval }\OperatorTok{-}\StringTok{ }\NormalTok{lis[i] ) }\OperatorTok{/}\StringTok{ }\NormalTok{( lis[j] }\OperatorTok{-}\StringTok{ }\NormalTok{lis[i] )}
\NormalTok{        val <-}\StringTok{ }\NormalTok{val }\OperatorTok{*}\StringTok{ }\NormalTok{nn}
\NormalTok{      \}}
\NormalTok{    \}}
\NormalTok{    tot <-}\StringTok{ }\NormalTok{tot }\OperatorTok{+}\StringTok{ }\NormalTok{val}
\NormalTok{  \}}
  \KeywordTok{return}\NormalTok{(tot)}
\NormalTok{\}}

\CommentTok{# esto, dada una lista, encuentra los 3 mejores puntos para lagrange, tomando los dos extremos y alguno intermedio}
\NormalTok{findMin <-}\StringTok{ }\ControlFlowTok{function}\NormalTok{(lisx, lisy, liminf, limsup)}
\NormalTok{\{}
  \KeywordTok{print}\NormalTok{(}\StringTok{"------------------------------------------------------"}\NormalTok{)}
  \KeywordTok{print}\NormalTok{(lisx)}
  \KeywordTok{cat}\NormalTok{(liminf,}\StringTok{"  "}\NormalTok{,limsup,}\StringTok{"}\CharTok{\textbackslash{}n}\StringTok{"}\NormalTok{)}
  \KeywordTok{print}\NormalTok{(lisx[liminf}\OperatorTok{:}\NormalTok{limsup])}
\NormalTok{  errmin <-}\StringTok{ }\DecValTok{0}
\NormalTok{  pt <-}\StringTok{ }\DecValTok{0}
\NormalTok{  prim <-}\StringTok{ }\OtherTok{TRUE}
  \ControlFlowTok{for}\NormalTok{( i }\ControlFlowTok{in}\NormalTok{ liminf}\OperatorTok{:}\NormalTok{limsup) ## se evitan los dos extremos}
\NormalTok{  \{}
    
    \KeywordTok{cat}\NormalTok{(}\StringTok{"-----------------------}\CharTok{\textbackslash{}n}\StringTok{"}\NormalTok{)}
    
\NormalTok{    valx <-}\StringTok{ }\NormalTok{lisx[liminf]}
\NormalTok{    valx <-}\StringTok{ }\KeywordTok{c}\NormalTok{(valx, lisx[i])}
\NormalTok{    valx <-}\StringTok{ }\KeywordTok{c}\NormalTok{(valx, lisx[limsup] )}
    
\NormalTok{    valy <-}\StringTok{ }\NormalTok{lisy[liminf]}
\NormalTok{    valy <-}\StringTok{ }\KeywordTok{c}\NormalTok{(valy,lisy[i])}
\NormalTok{    valy <-}\StringTok{ }\KeywordTok{c}\NormalTok{(valy, lisy[limsup] )}
    
\NormalTok{    sumerr <-}\StringTok{ }\DecValTok{0}    
    \ControlFlowTok{for}\NormalTok{( j }\ControlFlowTok{in}\NormalTok{ liminf}\OperatorTok{:}\NormalTok{limsup )}
\NormalTok{    \{}
\NormalTok{      sumerr <-}\StringTok{ }\NormalTok{sumerr }\OperatorTok{+}\StringTok{ }\KeywordTok{abs}\NormalTok{( }\KeywordTok{lagrange}\NormalTok{(lisx[j],valx , valy ) }\OperatorTok{-}\StringTok{ }\NormalTok{lisx[j])}
\NormalTok{    \}}
    
    \KeywordTok{cat}\NormalTok{(i,}\StringTok{" en pos "}\NormalTok{, sumerr,}\StringTok{"}\CharTok{\textbackslash{}n}\StringTok{"}\NormalTok{)}
    
    \ControlFlowTok{if}\NormalTok{ ( }\OperatorTok{!}\KeywordTok{is.nan}\NormalTok{(sumerr) }\OperatorTok{&&}\StringTok{ }\NormalTok{( prim }\OperatorTok{||}\StringTok{ }\NormalTok{sumerr }\OperatorTok{<}\StringTok{ }\NormalTok{errmin ) )}
\NormalTok{    \{}
\NormalTok{      errmin =}\StringTok{ }\NormalTok{sumerr}
      \KeywordTok{print}\NormalTok{(errmin)}
\NormalTok{      sumerr =}\StringTok{ }\DecValTok{2}
      \KeywordTok{print}\NormalTok{(errmin)}
\NormalTok{      pt =}\StringTok{ }\NormalTok{i }
\NormalTok{      prim =}\StringTok{ }\OtherTok{FALSE}
\NormalTok{    \}}
    
    
\NormalTok{  \}}
  \KeywordTok{cat}\NormalTok{ (}\StringTok{"final : "}\NormalTok{,errmin ,}\StringTok{"    "}\NormalTok{,pt,}\StringTok{"}\CharTok{\textbackslash{}n}\StringTok{"}\NormalTok{)}
  \KeywordTok{print}\NormalTok{(}\StringTok{"datos : "}\NormalTok{)}
\NormalTok{  nx =}\StringTok{ }\KeywordTok{c}\NormalTok{(lisx[liminf], lisx[pt] , lisx[limsup]  )}
\NormalTok{  ny =}\StringTok{ }\KeywordTok{c}\NormalTok{(lisy[liminf], lisy[pt] , lisy[limsup]  )}
  
\NormalTok{  valx <-}\StringTok{ }\NormalTok{lisx[liminf]}
\NormalTok{  valx <-}\StringTok{ }\KeywordTok{c}\NormalTok{(valx, lisx[pt])}
\NormalTok{  valx <-}\StringTok{ }\KeywordTok{c}\NormalTok{(valx, lisx[limsup] )}
  
\NormalTok{  valy <-}\StringTok{ }\NormalTok{lisy[liminf]}
\NormalTok{  valy <-}\StringTok{ }\KeywordTok{c}\NormalTok{(valy,lisy[pt])}
\NormalTok{  valy <-}\StringTok{ }\KeywordTok{c}\NormalTok{(valy, lisy[limsup] )}

\NormalTok{  eqn =}\StringTok{ }\ControlFlowTok{function}\NormalTok{(eval)\{ }\KeywordTok{lagrange}\NormalTok{(eval,nx , ny ) \}}
  \KeywordTok{graf}\NormalTok{(nx,ny,}\StringTok{"red"}\NormalTok{)}
  
  
\NormalTok{  eqn =}\StringTok{ }\ControlFlowTok{function}\NormalTok{(eval)\{ }\KeywordTok{lagrange}\NormalTok{(eval,nx , ny ) \}}
  
  \KeywordTok{curve}\NormalTok{(eqn, }\DataTypeTok{from=}\NormalTok{lisx[liminf], }\DataTypeTok{to=}\NormalTok{lisx[limsup],}\DataTypeTok{add=}\OtherTok{TRUE}\NormalTok{, }\DataTypeTok{col =} \StringTok{"red"}\NormalTok{)}
  
\NormalTok{\}}


\CommentTok{# reorganizacion de datos }
\NormalTok{x =}\StringTok{ }\KeywordTok{c}\NormalTok{(}\FloatTok{14.65}\NormalTok{, }\FloatTok{14.71}\NormalTok{, }\FloatTok{14.6}\NormalTok{, }\FloatTok{14.8}\NormalTok{, }\FloatTok{15.2}\NormalTok{, }\FloatTok{15.6}\NormalTok{, }\FloatTok{15.7}\NormalTok{, }\FloatTok{17.0}\NormalTok{, }\FloatTok{17.6}\NormalTok{, }\FloatTok{17.52}\NormalTok{, }\FloatTok{17.3}\NormalTok{, }
  \FloatTok{16.8}\NormalTok{, }\FloatTok{15.4}\NormalTok{, }\FloatTok{14.83}\NormalTok{, }\FloatTok{14.4}\NormalTok{, }\FloatTok{14.5}\NormalTok{, }
  \FloatTok{15.0}\NormalTok{, }\FloatTok{15.1}\NormalTok{, }\FloatTok{15.0}\NormalTok{, }\FloatTok{14.9}\NormalTok{, }\FloatTok{14.6}\NormalTok{, }\FloatTok{14.3}\NormalTok{, }\FloatTok{14.0}\NormalTok{, }\FloatTok{13.9}\NormalTok{, }\FloatTok{13.8}\NormalTok{, }\FloatTok{13.5}\NormalTok{, }\FloatTok{13.1}\NormalTok{, }\FloatTok{13.0}\NormalTok{, }
  \FloatTok{13.3}\NormalTok{, }\FloatTok{13.2}\NormalTok{, }\FloatTok{13.1}\NormalTok{, }\FloatTok{12.9}\NormalTok{, }\FloatTok{12.4}\NormalTok{, }\FloatTok{11.9}\NormalTok{, }\FloatTok{11.7}\NormalTok{, }\FloatTok{11.6}\NormalTok{, }\FloatTok{11.3}\NormalTok{, }\FloatTok{10.9}\NormalTok{, }
  \FloatTok{10.7}\NormalTok{, }\FloatTok{10.6}\NormalTok{, }\FloatTok{10.6}\NormalTok{, }\FloatTok{10.1}\NormalTok{, }\FloatTok{9.7}\NormalTok{, }\FloatTok{9.4}\NormalTok{, }\FloatTok{9.3}\NormalTok{, }\FloatTok{9.6}\NormalTok{, }\FloatTok{9.9}\NormalTok{, }\FloatTok{10.1}\NormalTok{, }\FloatTok{10.2}\NormalTok{, }\FloatTok{10.3}\NormalTok{,  }\FloatTok{10.0}\NormalTok{, }\FloatTok{9.5}\NormalTok{, }
  \FloatTok{8.6}\NormalTok{, }\FloatTok{7.5}\NormalTok{, }\FloatTok{7.0}\NormalTok{, }\FloatTok{6.7}\NormalTok{, }\FloatTok{6.6}\NormalTok{, }\FloatTok{7.7}\NormalTok{, }
  \FloatTok{8.0}\NormalTok{, }\FloatTok{8.1}\NormalTok{, }\FloatTok{8.4}\NormalTok{, }\FloatTok{9.2}\NormalTok{, }\FloatTok{9.3}\NormalTok{, }\DecValTok{10}\NormalTok{, }\FloatTok{10.2}\NormalTok{, }\FloatTok{10.3}\NormalTok{)}

\NormalTok{y =}\StringTok{ }\KeywordTok{c}\NormalTok{(}\FloatTok{14.7}\NormalTok{, }\FloatTok{14.33}\NormalTok{, }\FloatTok{13.4}\NormalTok{, }\FloatTok{12.33}\NormalTok{, }\FloatTok{11.0}\NormalTok{, }\FloatTok{10.5}\NormalTok{, }\FloatTok{10.22}\NormalTok{, }\FloatTok{8.2}\NormalTok{, }\FloatTok{7.1}\NormalTok{, }\FloatTok{6.7}\NormalTok{, }\FloatTok{6.6}\NormalTok{, }
  \FloatTok{6.8}\NormalTok{, }\FloatTok{8.3}\NormalTok{, }\FloatTok{8.8}\NormalTok{, }\FloatTok{9.3}\NormalTok{, }\FloatTok{8.8}\NormalTok{, }
  \FloatTok{6.3}\NormalTok{, }\FloatTok{5.5}\NormalTok{, }\FloatTok{5.0}\NormalTok{, }\FloatTok{4.7}\NormalTok{, }\FloatTok{4.6}\NormalTok{, }\FloatTok{4.5}\NormalTok{, }\FloatTok{4.9}\NormalTok{, }\FloatTok{5.4}\NormalTok{, }\FloatTok{5.8}\NormalTok{, }\FloatTok{6.9}\NormalTok{, }\FloatTok{8.2}\NormalTok{, }\FloatTok{7.6}\NormalTok{, }
  \FloatTok{5.8}\NormalTok{, }\FloatTok{4.5}\NormalTok{, }\FloatTok{4.3}\NormalTok{, }\FloatTok{3.9}\NormalTok{, }\FloatTok{4.2}\NormalTok{, }\FloatTok{5.7}\NormalTok{, }\FloatTok{7.0}\NormalTok{, }\FloatTok{7.9}\NormalTok{, }\FloatTok{8.2}\NormalTok{, }\FloatTok{7.3}\NormalTok{, }
  \FloatTok{6.7}\NormalTok{, }\FloatTok{5.5}\NormalTok{, }\FloatTok{5.1}\NormalTok{, }\FloatTok{4.6}\NormalTok{, }\FloatTok{4.7}\NormalTok{, }\FloatTok{5.0}\NormalTok{, }\FloatTok{5.5}\NormalTok{, }\FloatTok{7.2}\NormalTok{, }\FloatTok{7.8}\NormalTok{, }\FloatTok{8.6}\NormalTok{, }\FloatTok{9.4}\NormalTok{, }\FloatTok{10.0}\NormalTok{,  }\FloatTok{10.7}\NormalTok{, }\FloatTok{11.0}\NormalTok{, }
  \FloatTok{9.9}\NormalTok{, }\FloatTok{9.0}\NormalTok{, }\FloatTok{9.1}\NormalTok{, }\FloatTok{9.3}\NormalTok{, }\FloatTok{9.7}\NormalTok{, }\FloatTok{11.7}\NormalTok{, }
  \FloatTok{12.3}\NormalTok{, }\FloatTok{12.5}\NormalTok{, }\FloatTok{13.0}\NormalTok{, }\FloatTok{13.91}\NormalTok{, }\FloatTok{14.9}\NormalTok{, }\DecValTok{16}\NormalTok{, }\FloatTok{16.4}\NormalTok{, }\FloatTok{16.8}\NormalTok{)}

\KeywordTok{plot}\NormalTok{(x,y,}\DataTypeTok{type=}\StringTok{"l"}\NormalTok{,}\DataTypeTok{asp=}\DecValTok{1}\NormalTok{)}
\NormalTok{## de cada uno de estos cortes deberian terminar quedando unos 3 puntos }
\NormalTok{c1 =}\StringTok{ }\DecValTok{10}
\NormalTok{c2 =}\StringTok{ }\DecValTok{15}
\NormalTok{c3 =}\StringTok{ }\DecValTok{27}
\NormalTok{c4 =}\StringTok{ }\DecValTok{37}
\NormalTok{c5 =}\StringTok{ }\DecValTok{50}
\NormalTok{c6 =}\StringTok{ }\DecValTok{56}
\NormalTok{c7 =}\StringTok{ }\KeywordTok{length}\NormalTok{(x)}

\KeywordTok{print}\NormalTok{(}\StringTok{"1"}\NormalTok{)}
\end{Highlighting}
\end{Shaded}

\begin{verbatim}
## [1] "1"
\end{verbatim}

\begin{Shaded}
\begin{Highlighting}[]
\KeywordTok{findMin}\NormalTok{(x,y,}\DecValTok{1}\NormalTok{,c1)}
\end{Highlighting}
\end{Shaded}

\begin{verbatim}
## [1] "------------------------------------------------------"
##  [1] 14.65 14.71 14.60 14.80 15.20 15.60 15.70 17.00 17.60 17.52 17.30
## [12] 16.80 15.40 14.83 14.40 14.50 15.00 15.10 15.00 14.90 14.60 14.30
## [23] 14.00 13.90 13.80 13.50 13.10 13.00 13.30 13.20 13.10 12.90 12.40
## [34] 11.90 11.70 11.60 11.30 10.90 10.70 10.60 10.60 10.10  9.70  9.40
## [45]  9.30  9.60  9.90 10.10 10.20 10.30 10.00  9.50  8.60  7.50  7.00
## [56]  6.70  6.60  7.70  8.00  8.10  8.40  9.20  9.30 10.00 10.20 10.30
## 1    10 
##  [1] 14.65 14.71 14.60 14.80 15.20 15.60 15.70 17.00 17.60 17.52
## -----------------------
## 1  en pos  NaN 
## -----------------------
## 2  en pos  49.36703 
## [1] 49.36703
## [1] 49.36703
## -----------------------
## 3  en pos  73.59968 
## -----------------------
## 4  en pos  73.433 
## -----------------------
## 5  en pos  52.69747 
## -----------------------
## 6  en pos  47.00352 
## [1] 47.00352
## [1] 47.00352
## -----------------------
## 7  en pos  46.74765 
## [1] 46.74765
## [1] 46.74765
## -----------------------
## 8  en pos  41.00846 
## [1] 41.00846
## [1] 41.00846
## -----------------------
## 9  en pos  59.02478 
## -----------------------
## 10  en pos  NaN 
## final :  41.00846      8 
## [1] "datos : "
\end{verbatim}

\begin{Shaded}
\begin{Highlighting}[]
\KeywordTok{print}\NormalTok{(}\StringTok{"2"}\NormalTok{)}
\end{Highlighting}
\end{Shaded}

\begin{verbatim}
## [1] "2"
\end{verbatim}

\begin{Shaded}
\begin{Highlighting}[]
\NormalTok{(}\KeywordTok{findMin}\NormalTok{(x,y,c1,c2) )}
\end{Highlighting}
\end{Shaded}

\begin{verbatim}
## [1] "------------------------------------------------------"
##  [1] 14.65 14.71 14.60 14.80 15.20 15.60 15.70 17.00 17.60 17.52 17.30
## [12] 16.80 15.40 14.83 14.40 14.50 15.00 15.10 15.00 14.90 14.60 14.30
## [23] 14.00 13.90 13.80 13.50 13.10 13.00 13.30 13.20 13.10 12.90 12.40
## [34] 11.90 11.70 11.60 11.30 10.90 10.70 10.60 10.60 10.10  9.70  9.40
## [45]  9.30  9.60  9.90 10.10 10.20 10.30 10.00  9.50  8.60  7.50  7.00
## [56]  6.70  6.60  7.70  8.00  8.10  8.40  9.20  9.30 10.00 10.20 10.30
## 10    15 
## [1] 17.52 17.30 16.80 15.40 14.83 14.40
## -----------------------
## 10  en pos  NaN 
## -----------------------
## 11  en pos  51.16423 
## [1] 51.16423
## [1] 51.16423
## -----------------------
## 12  en pos  50.29106 
## [1] 50.29106
## [1] 50.29106
## -----------------------
## 13  en pos  49.10194 
## [1] 49.10194
## [1] 49.10194
## -----------------------
## 14  en pos  49.34942 
## -----------------------
## 15  en pos  NaN 
## final :  49.10194      13 
## [1] "datos : "
\end{verbatim}

\begin{verbatim}
## $x
##   [1] 17.5200 17.4888 17.4576 17.4264 17.3952 17.3640 17.3328 17.3016
##   [9] 17.2704 17.2392 17.2080 17.1768 17.1456 17.1144 17.0832 17.0520
##  [17] 17.0208 16.9896 16.9584 16.9272 16.8960 16.8648 16.8336 16.8024
##  [25] 16.7712 16.7400 16.7088 16.6776 16.6464 16.6152 16.5840 16.5528
##  [33] 16.5216 16.4904 16.4592 16.4280 16.3968 16.3656 16.3344 16.3032
##  [41] 16.2720 16.2408 16.2096 16.1784 16.1472 16.1160 16.0848 16.0536
##  [49] 16.0224 15.9912 15.9600 15.9288 15.8976 15.8664 15.8352 15.8040
##  [57] 15.7728 15.7416 15.7104 15.6792 15.6480 15.6168 15.5856 15.5544
##  [65] 15.5232 15.4920 15.4608 15.4296 15.3984 15.3672 15.3360 15.3048
##  [73] 15.2736 15.2424 15.2112 15.1800 15.1488 15.1176 15.0864 15.0552
##  [81] 15.0240 14.9928 14.9616 14.9304 14.8992 14.8680 14.8368 14.8056
##  [89] 14.7744 14.7432 14.7120 14.6808 14.6496 14.6184 14.5872 14.5560
##  [97] 14.5248 14.4936 14.4624 14.4312 14.4000
## 
## $y
##   [1] 6.700000 6.718424 6.737000 6.755730 6.774613 6.793649 6.812838
##   [8] 6.832180 6.851675 6.871323 6.891125 6.911079 6.931186 6.951446
##  [15] 6.971860 6.992426 7.013146 7.034019 7.055044 7.076223 7.097555
##  [22] 7.119040 7.140677 7.162468 7.184412 7.206509 7.228760 7.251163
##  [29] 7.273719 7.296428 7.319291 7.342306 7.365474 7.388796 7.412270
##  [36] 7.435898 7.459679 7.483613 7.507699 7.531939 7.556332 7.580878
##  [43] 7.605577 7.630429 7.655434 7.680592 7.705904 7.731368 7.756985
##  [50] 7.782756 7.808679 7.834756 7.860985 7.887368 7.913904 7.940592
##  [57] 7.967434 7.994429 8.021577 8.048878 8.076332 8.103939 8.131699
##  [64] 8.159613 8.187679 8.215898 8.244270 8.272796 8.301474 8.330306
##  [71] 8.359291 8.388428 8.417719 8.447163 8.476760 8.506509 8.536412
##  [78] 8.566468 8.596677 8.627040 8.657555 8.688223 8.719044 8.750019
##  [85] 8.781146 8.812426 8.843860 8.875446 8.907186 8.939079 8.971125
##  [92] 9.003323 9.035675 9.068180 9.100838 9.133649 9.166613 9.199730
##  [99] 9.233000 9.266424 9.300000
\end{verbatim}

\begin{Shaded}
\begin{Highlighting}[]
\KeywordTok{print}\NormalTok{(}\StringTok{"3"}\NormalTok{)}
\end{Highlighting}
\end{Shaded}

\begin{verbatim}
## [1] "3"
\end{verbatim}

\begin{Shaded}
\begin{Highlighting}[]
\NormalTok{(}\KeywordTok{findMin}\NormalTok{(x,y,c2,c3) )}
\end{Highlighting}
\end{Shaded}

\begin{verbatim}
## [1] "------------------------------------------------------"
##  [1] 14.65 14.71 14.60 14.80 15.20 15.60 15.70 17.00 17.60 17.52 17.30
## [12] 16.80 15.40 14.83 14.40 14.50 15.00 15.10 15.00 14.90 14.60 14.30
## [23] 14.00 13.90 13.80 13.50 13.10 13.00 13.30 13.20 13.10 12.90 12.40
## [34] 11.90 11.70 11.60 11.30 10.90 10.70 10.60 10.60 10.10  9.70  9.40
## [45]  9.30  9.60  9.90 10.10 10.20 10.30 10.00  9.50  8.60  7.50  7.00
## [56]  6.70  6.60  7.70  8.00  8.10  8.40  9.20  9.30 10.00 10.20 10.30
## 15    27 
##  [1] 14.4 14.5 15.0 15.1 15.0 14.9 14.6 14.3 14.0 13.9 13.8 13.5 13.1
## -----------------------
## 15  en pos  NaN 
## -----------------------
## 16  en pos  80.16154 
## [1] 80.16154
## [1] 80.16154
## -----------------------
## 17  en pos  76.46923 
## [1] 76.46923
## [1] 76.46923
## -----------------------
## 18  en pos  76.67231 
## -----------------------
## 19  en pos  80.30081 
## -----------------------
## 20  en pos  84.88359 
## -----------------------
## 21  en pos  130.5954 
## -----------------------
## 22  en pos  266.5897 
## -----------------------
## 23  en pos  90.02821 
## -----------------------
## 24  en pos  75.00769 
## [1] 75.00769
## [1] 75.00769
## -----------------------
## 25  en pos  65.91465 
## [1] 65.91465
## [1] 65.91465
## -----------------------
## 26  en pos  53.16667 
## [1] 53.16667
## [1] 53.16667
## -----------------------
## 27  en pos  NaN 
## final :  53.16667      26 
## [1] "datos : "
\end{verbatim}

\begin{verbatim}
## $x
##   [1] 14.400 14.387 14.374 14.361 14.348 14.335 14.322 14.309 14.296 14.283
##  [11] 14.270 14.257 14.244 14.231 14.218 14.205 14.192 14.179 14.166 14.153
##  [21] 14.140 14.127 14.114 14.101 14.088 14.075 14.062 14.049 14.036 14.023
##  [31] 14.010 13.997 13.984 13.971 13.958 13.945 13.932 13.919 13.906 13.893
##  [41] 13.880 13.867 13.854 13.841 13.828 13.815 13.802 13.789 13.776 13.763
##  [51] 13.750 13.737 13.724 13.711 13.698 13.685 13.672 13.659 13.646 13.633
##  [61] 13.620 13.607 13.594 13.581 13.568 13.555 13.542 13.529 13.516 13.503
##  [71] 13.490 13.477 13.464 13.451 13.438 13.425 13.412 13.399 13.386 13.373
##  [81] 13.360 13.347 13.334 13.321 13.308 13.295 13.282 13.269 13.256 13.243
##  [91] 13.230 13.217 13.204 13.191 13.178 13.165 13.152 13.139 13.126 13.113
## [101] 13.100
## 
## $y
##   [1] 9.300000 9.212853 9.127243 9.043173 8.960640 8.879646 8.800190
##   [8] 8.722273 8.645893 8.571052 8.497750 8.425986 8.355760 8.287072
##  [15] 8.219923 8.154312 8.090240 8.027706 7.966710 7.907253 7.849333
##  [22] 7.792953 7.738110 7.684806 7.633040 7.582812 7.534123 7.486972
##  [29] 7.441360 7.397286 7.354750 7.313752 7.274293 7.236373 7.199990
##  [36] 7.165146 7.131840 7.100073 7.069843 7.041153 7.014000 6.988386
##  [43] 6.964310 6.941773 6.920773 6.901313 6.883390 6.867006 6.852160
##  [50] 6.838853 6.827083 6.816853 6.808160 6.801006 6.795390 6.791313
##  [57] 6.788773 6.787772 6.788310 6.790386 6.794000 6.799153 6.805843
##  [64] 6.814073 6.823840 6.835146 6.847990 6.862373 6.878293 6.895753
##  [71] 6.914750 6.935286 6.957360 6.980972 7.006123 7.032812 7.061040
##  [78] 7.090806 7.122110 7.154953 7.189333 7.225253 7.262710 7.301706
##  [85] 7.342240 7.384313 7.427923 7.473072 7.519760 7.567986 7.617750
##  [92] 7.669052 7.721893 7.776273 7.832190 7.889646 7.948640 8.009173
##  [99] 8.071243 8.134853 8.200000
\end{verbatim}

\begin{Shaded}
\begin{Highlighting}[]
\KeywordTok{print}\NormalTok{(}\StringTok{"4"}\NormalTok{)}
\end{Highlighting}
\end{Shaded}

\begin{verbatim}
## [1] "4"
\end{verbatim}

\begin{Shaded}
\begin{Highlighting}[]
\NormalTok{(}\KeywordTok{findMin}\NormalTok{(x,y,c3,c4) ) }
\end{Highlighting}
\end{Shaded}

\begin{verbatim}
## [1] "------------------------------------------------------"
##  [1] 14.65 14.71 14.60 14.80 15.20 15.60 15.70 17.00 17.60 17.52 17.30
## [12] 16.80 15.40 14.83 14.40 14.50 15.00 15.10 15.00 14.90 14.60 14.30
## [23] 14.00 13.90 13.80 13.50 13.10 13.00 13.30 13.20 13.10 12.90 12.40
## [34] 11.90 11.70 11.60 11.30 10.90 10.70 10.60 10.60 10.10  9.70  9.40
## [45]  9.30  9.60  9.90 10.10 10.20 10.30 10.00  9.50  8.60  7.50  7.00
## [56]  6.70  6.60  7.70  8.00  8.10  8.40  9.20  9.30 10.00 10.20 10.30
## 27    37 
##  [1] 13.1 13.0 13.3 13.2 13.1 12.9 12.4 11.9 11.7 11.6 11.3
## -----------------------
## 27  en pos  NaN 
## -----------------------
## 28  en pos  55.77059 
## [1] 55.77059
## [1] 55.77059
## -----------------------
## 29  en pos  34.98 
## [1] 34.98
## [1] 34.98
## -----------------------
## 30  en pos  71.39474 
## -----------------------
## 31  en pos  NaN 
## -----------------------
## 32  en pos  80.1 
## -----------------------
## 33  en pos  59.76753 
## -----------------------
## 34  en pos  55.63333 
## -----------------------
## 35  en pos  52.44286 
## -----------------------
## 36  en pos  48.9 
## -----------------------
## 37  en pos  NaN 
## final :  34.98      29 
## [1] "datos : "
\end{verbatim}

\begin{verbatim}
## $x
##   [1] 13.100 13.082 13.064 13.046 13.028 13.010 12.992 12.974 12.956 12.938
##  [11] 12.920 12.902 12.884 12.866 12.848 12.830 12.812 12.794 12.776 12.758
##  [21] 12.740 12.722 12.704 12.686 12.668 12.650 12.632 12.614 12.596 12.578
##  [31] 12.560 12.542 12.524 12.506 12.488 12.470 12.452 12.434 12.416 12.398
##  [41] 12.380 12.362 12.344 12.326 12.308 12.290 12.272 12.254 12.236 12.218
##  [51] 12.200 12.182 12.164 12.146 12.128 12.110 12.092 12.074 12.056 12.038
##  [61] 12.020 12.002 11.984 11.966 11.948 11.930 11.912 11.894 11.876 11.858
##  [71] 11.840 11.822 11.804 11.786 11.768 11.750 11.732 11.714 11.696 11.678
##  [81] 11.660 11.642 11.624 11.606 11.588 11.570 11.552 11.534 11.516 11.498
##  [91] 11.480 11.462 11.444 11.426 11.408 11.390 11.372 11.354 11.336 11.318
## [101] 11.300
## 
## $y
##   [1]  8.200000  8.392456  8.581024  8.765704  8.946496  9.123400  9.296416
##   [8]  9.465544  9.630784  9.792136  9.949600 10.103176 10.252864 10.398664
##  [15] 10.540576 10.678600 10.812736 10.942984 11.069344 11.191816 11.310400
##  [22] 11.425096 11.535904 11.642824 11.745856 11.845000 11.940256 12.031624
##  [29] 12.119104 12.202696 12.282400 12.358216 12.430144 12.498184 12.562336
##  [36] 12.622600 12.678976 12.731464 12.780064 12.824776 12.865600 12.902536
##  [43] 12.935584 12.964744 12.990016 13.011400 13.028896 13.042504 13.052224
##  [50] 13.058056 13.060000 13.058056 13.052224 13.042504 13.028896 13.011400
##  [57] 12.990016 12.964744 12.935584 12.902536 12.865600 12.824776 12.780064
##  [64] 12.731464 12.678976 12.622600 12.562336 12.498184 12.430144 12.358216
##  [71] 12.282400 12.202696 12.119104 12.031624 11.940256 11.845000 11.745856
##  [78] 11.642824 11.535904 11.425096 11.310400 11.191816 11.069344 10.942984
##  [85] 10.812736 10.678600 10.540576 10.398664 10.252864 10.103176  9.949600
##  [92]  9.792136  9.630784  9.465544  9.296416  9.123400  8.946496  8.765704
##  [99]  8.581024  8.392456  8.200000
\end{verbatim}

\begin{Shaded}
\begin{Highlighting}[]
\KeywordTok{print}\NormalTok{(}\StringTok{"5"}\NormalTok{)}
\end{Highlighting}
\end{Shaded}

\begin{verbatim}
## [1] "5"
\end{verbatim}

\begin{Shaded}
\begin{Highlighting}[]
\NormalTok{(}\KeywordTok{findMin}\NormalTok{(x,y,c4,c5) )}
\end{Highlighting}
\end{Shaded}

\begin{verbatim}
## [1] "------------------------------------------------------"
##  [1] 14.65 14.71 14.60 14.80 15.20 15.60 15.70 17.00 17.60 17.52 17.30
## [12] 16.80 15.40 14.83 14.40 14.50 15.00 15.10 15.00 14.90 14.60 14.30
## [23] 14.00 13.90 13.80 13.50 13.10 13.00 13.30 13.20 13.10 12.90 12.40
## [34] 11.90 11.70 11.60 11.30 10.90 10.70 10.60 10.60 10.10  9.70  9.40
## [45]  9.30  9.60  9.90 10.10 10.20 10.30 10.00  9.50  8.60  7.50  7.00
## [56]  6.70  6.60  7.70  8.00  8.10  8.40  9.20  9.30 10.00 10.20 10.30
## 37    50 
##  [1] 11.3 10.9 10.7 10.6 10.6 10.1  9.7  9.4  9.3  9.6  9.9 10.1 10.2 10.3
## -----------------------
## 37  en pos  NaN 
## -----------------------
## 38  en pos  71.5525 
## [1] 71.5525
## [1] 71.5525
## -----------------------
## 39  en pos  103.1925 
## -----------------------
## 40  en pos  167.32 
## -----------------------
## 41  en pos  182.3867 
## -----------------------
## 42  en pos  178.48 
## -----------------------
## 43  en pos  41.97854 
## [1] 41.97854
## [1] 41.97854
## -----------------------
## 44  en pos  23.65392 
## [1] 23.65392
## [1] 23.65392
## -----------------------
## 45  en pos  19.2465 
## [1] 19.2465
## [1] 19.2465
## -----------------------
## 46  en pos  20.84588 
## -----------------------
## 47  en pos  31.85929 
## -----------------------
## 48  en pos  47.08667 
## -----------------------
## 49  en pos  45.28545 
## -----------------------
## 50  en pos  NaN 
## final :  19.2465      45 
## [1] "datos : "
\end{verbatim}

\begin{verbatim}
## $x
##   [1] 11.30 11.29 11.28 11.27 11.26 11.25 11.24 11.23 11.22 11.21 11.20
##  [12] 11.19 11.18 11.17 11.16 11.15 11.14 11.13 11.12 11.11 11.10 11.09
##  [23] 11.08 11.07 11.06 11.05 11.04 11.03 11.02 11.01 11.00 10.99 10.98
##  [34] 10.97 10.96 10.95 10.94 10.93 10.92 10.91 10.90 10.89 10.88 10.87
##  [45] 10.86 10.85 10.84 10.83 10.82 10.81 10.80 10.79 10.78 10.77 10.76
##  [56] 10.75 10.74 10.73 10.72 10.71 10.70 10.69 10.68 10.67 10.66 10.65
##  [67] 10.64 10.63 10.62 10.61 10.60 10.59 10.58 10.57 10.56 10.55 10.54
##  [78] 10.53 10.52 10.51 10.50 10.49 10.48 10.47 10.46 10.45 10.44 10.43
##  [89] 10.42 10.41 10.40 10.39 10.38 10.37 10.36 10.35 10.34 10.33 10.32
## [100] 10.31 10.30
## 
## $y
##   [1]  8.200000  8.249185  8.297740  8.345665  8.392960  8.439625  8.485660
##   [8]  8.531065  8.575840  8.619985  8.663500  8.706385  8.748640  8.790265
##  [15]  8.831260  8.871625  8.911360  8.950465  8.988940  9.026785  9.064000
##  [22]  9.100585  9.136540  9.171865  9.206560  9.240625  9.274060  9.306865
##  [29]  9.339040  9.370585  9.401500  9.431785  9.461440  9.490465  9.518860
##  [36]  9.546625  9.573760  9.600265  9.626140  9.651385  9.676000  9.699985
##  [43]  9.723340  9.746065  9.768160  9.789625  9.810460  9.830665  9.850240
##  [50]  9.869185  9.887500  9.905185  9.922240  9.938665  9.954460  9.969625
##  [57]  9.984160  9.998065 10.011340 10.023985 10.036000 10.047385 10.058140
##  [64] 10.068265 10.077760 10.086625 10.094860 10.102465 10.109440 10.115785
##  [71] 10.121500 10.126585 10.131040 10.134865 10.138060 10.140625 10.142560
##  [78] 10.143865 10.144540 10.144585 10.144000 10.142785 10.140940 10.138465
##  [85] 10.135360 10.131625 10.127260 10.122265 10.116640 10.110385 10.103500
##  [92] 10.095985 10.087840 10.079065 10.069660 10.059625 10.048960 10.037665
##  [99] 10.025740 10.013185 10.000000
\end{verbatim}

\begin{Shaded}
\begin{Highlighting}[]
\KeywordTok{print}\NormalTok{(}\StringTok{"6"}\NormalTok{)}
\end{Highlighting}
\end{Shaded}

\begin{verbatim}
## [1] "6"
\end{verbatim}

\begin{Shaded}
\begin{Highlighting}[]
\NormalTok{(}\KeywordTok{findMin}\NormalTok{(x,y,c5,c6) )}
\end{Highlighting}
\end{Shaded}

\begin{verbatim}
## [1] "------------------------------------------------------"
##  [1] 14.65 14.71 14.60 14.80 15.20 15.60 15.70 17.00 17.60 17.52 17.30
## [12] 16.80 15.40 14.83 14.40 14.50 15.00 15.10 15.00 14.90 14.60 14.30
## [23] 14.00 13.90 13.80 13.50 13.10 13.00 13.30 13.20 13.10 12.90 12.40
## [34] 11.90 11.70 11.60 11.30 10.90 10.70 10.60 10.60 10.10  9.70  9.40
## [45]  9.30  9.60  9.90 10.10 10.20 10.30 10.00  9.50  8.60  7.50  7.00
## [56]  6.70  6.60  7.70  8.00  8.10  8.40  9.20  9.30 10.00 10.20 10.30
## 50    56 
## [1] 10.3 10.0  9.5  8.6  7.5  7.0  6.7
## -----------------------
## 50  en pos  NaN 
## -----------------------
## 51  en pos  15.99192 
## [1] 15.99192
## [1] 15.99192
## -----------------------
## 52  en pos  13.56825 
## [1] 13.56825
## [1] 13.56825
## -----------------------
## 53  en pos  9.261111 
## [1] 9.261111
## [1] 9.261111
## -----------------------
## 54  en pos  7.340327 
## [1] 7.340327
## [1] 7.340327
## -----------------------
## 55  en pos  7.154377 
## [1] 7.154377
## [1] 7.154377
## -----------------------
## 56  en pos  NaN 
## final :  7.154377      55 
## [1] "datos : "
\end{verbatim}

\begin{verbatim}
## $x
##   [1] 10.300 10.264 10.228 10.192 10.156 10.120 10.084 10.048 10.012  9.976
##  [11]  9.940  9.904  9.868  9.832  9.796  9.760  9.724  9.688  9.652  9.616
##  [21]  9.580  9.544  9.508  9.472  9.436  9.400  9.364  9.328  9.292  9.256
##  [31]  9.220  9.184  9.148  9.112  9.076  9.040  9.004  8.968  8.932  8.896
##  [41]  8.860  8.824  8.788  8.752  8.716  8.680  8.644  8.608  8.572  8.536
##  [51]  8.500  8.464  8.428  8.392  8.356  8.320  8.284  8.248  8.212  8.176
##  [61]  8.140  8.104  8.068  8.032  7.996  7.960  7.924  7.888  7.852  7.816
##  [71]  7.780  7.744  7.708  7.672  7.636  7.600  7.564  7.528  7.492  7.456
##  [81]  7.420  7.384  7.348  7.312  7.276  7.240  7.204  7.168  7.132  7.096
##  [91]  7.060  7.024  6.988  6.952  6.916  6.880  6.844  6.808  6.772  6.736
## [101]  6.700
## 
## $y
##   [1] 10.000000  9.959520  9.919716  9.880589  9.842138  9.804364  9.767265
##   [8]  9.730844  9.695098  9.660029  9.625636  9.591920  9.558880  9.526516
##  [15]  9.494829  9.463818  9.433484  9.403825  9.374844  9.346538  9.318909
##  [22]  9.291956  9.265680  9.240080  9.215156  9.190909  9.167338  9.144444
##  [29]  9.122225  9.100684  9.079818  9.059629  9.040116  9.021280  9.003120
##  [36]  8.985636  8.968829  8.952698  8.937244  8.922465  8.908364  8.894938
##  [43]  8.882189  8.870116  8.858720  8.848000  8.837956  8.828589  8.819898
##  [50]  8.811884  8.804545  8.797884  8.791898  8.786589  8.781956  8.778000
##  [57]  8.774720  8.772116  8.770189  8.768938  8.768364  8.768465  8.769244
##  [64]  8.770698  8.772829  8.775636  8.779120  8.783280  8.788116  8.793629
##  [71]  8.799818  8.806684  8.814225  8.822444  8.831338  8.840909  8.851156
##  [78]  8.862080  8.873680  8.885956  8.898909  8.912538  8.926844  8.941825
##  [85]  8.957484  8.973818  8.990829  9.008516  9.026880  9.045920  9.065636
##  [92]  9.086029  9.107098  9.128844  9.151265  9.174364  9.198138  9.222589
##  [99]  9.247716  9.273520  9.300000
\end{verbatim}

\begin{Shaded}
\begin{Highlighting}[]
\KeywordTok{print}\NormalTok{(}\StringTok{"7"}\NormalTok{)}
\end{Highlighting}
\end{Shaded}

\begin{verbatim}
## [1] "7"
\end{verbatim}

\begin{Shaded}
\begin{Highlighting}[]
\NormalTok{(}\KeywordTok{findMin}\NormalTok{(x,y,c6,c7) )}
\end{Highlighting}
\end{Shaded}

\includegraphics{Documento_notebook_files/figure-latex/unnamed-chunk-4-1.pdf}

\begin{verbatim}
## [1] "------------------------------------------------------"
##  [1] 14.65 14.71 14.60 14.80 15.20 15.60 15.70 17.00 17.60 17.52 17.30
## [12] 16.80 15.40 14.83 14.40 14.50 15.00 15.10 15.00 14.90 14.60 14.30
## [23] 14.00 13.90 13.80 13.50 13.10 13.00 13.30 13.20 13.10 12.90 12.40
## [34] 11.90 11.70 11.60 11.30 10.90 10.70 10.60 10.60 10.10  9.70  9.40
## [45]  9.30  9.60  9.90 10.10 10.20 10.30 10.00  9.50  8.60  7.50  7.00
## [56]  6.70  6.60  7.70  8.00  8.10  8.40  9.20  9.30 10.00 10.20 10.30
## 56    66 
##  [1]  6.7  6.6  7.7  8.0  8.1  8.4  9.2  9.3 10.0 10.2 10.3
## -----------------------
## 56  en pos  NaN 
## -----------------------
## 57  en pos  27.80766 
## [1] 27.80766
## [1] 27.80766
## -----------------------
## 58  en pos  53.35244 
## -----------------------
## 59  en pos  52.91065 
## -----------------------
## 60  en pos  52.80942 
## -----------------------
## 61  en pos  52.02647 
## -----------------------
## 62  en pos  47.1691 
## -----------------------
## 63  en pos  52.41808 
## -----------------------
## 64  en pos  47.91263 
## -----------------------
## 65  en pos  41.15571 
## -----------------------
## 66  en pos  NaN 
## final :  27.80766      57 
## [1] "datos : "
\end{verbatim}

\begin{verbatim}
## $x
##   [1]  6.700  6.736  6.772  6.808  6.844  6.880  6.916  6.952  6.988  7.024
##  [11]  7.060  7.096  7.132  7.168  7.204  7.240  7.276  7.312  7.348  7.384
##  [21]  7.420  7.456  7.492  7.528  7.564  7.600  7.636  7.672  7.708  7.744
##  [31]  7.780  7.816  7.852  7.888  7.924  7.960  7.996  8.032  8.068  8.104
##  [41]  8.140  8.176  8.212  8.248  8.284  8.320  8.356  8.392  8.428  8.464
##  [51]  8.500  8.536  8.572  8.608  8.644  8.680  8.716  8.752  8.788  8.824
##  [61]  8.860  8.896  8.932  8.968  9.004  9.040  9.076  9.112  9.148  9.184
##  [71]  9.220  9.256  9.292  9.328  9.364  9.400  9.436  9.472  9.508  9.544
##  [81]  9.580  9.616  9.652  9.688  9.724  9.760  9.796  9.832  9.868  9.904
##  [91]  9.940  9.976 10.012 10.048 10.084 10.120 10.156 10.192 10.228 10.264
## [101] 10.300
## 
## $y
##   [1]  9.300000  9.164050  9.032361  8.904934  8.781769  8.662865  8.548223
##   [8]  8.437842  8.331723  8.229866  8.132270  8.038936  7.949864  7.865053
##  [15]  7.784504  7.708216  7.636190  7.568426  7.504923  7.445682  7.390703
##  [22]  7.339985  7.293529  7.251334  7.213401  7.179730  7.150320  7.125172
##  [29]  7.104285  7.087661  7.075297  7.067196  7.063356  7.063777  7.068461
##  [36]  7.077405  7.090612  7.108080  7.129810  7.155801  7.186054  7.220569
##  [43]  7.259345  7.302383  7.349682  7.401243  7.457066  7.517150  7.581496
##  [50]  7.650104  7.722973  7.800104  7.881496  7.967150  8.057066  8.151243
##  [57]  8.249682  8.352383  8.459345  8.570569  8.686054  8.805801  8.929810
##  [64]  9.058080  9.190612  9.327405  9.468461  9.613777  9.763356  9.917196
##  [71] 10.075297 10.237661 10.404285 10.575172 10.750320 10.929730 11.113401
##  [78] 11.301334 11.493529 11.689985 11.890703 12.095682 12.304923 12.518426
##  [85] 12.736190 12.958216 13.184504 13.415053 13.649864 13.888936 14.132270
##  [92] 14.379866 14.631723 14.887842 15.148223 15.412865 15.681769 15.954934
##  [99] 16.232361 16.514050 16.800000
\end{verbatim}

\begin{Shaded}
\begin{Highlighting}[]
\KeywordTok{print}\NormalTok{(}\StringTok{"8"}\NormalTok{)}
\end{Highlighting}
\end{Shaded}

\begin{verbatim}
## [1] "8"
\end{verbatim}

\subsubsection{Versión 2}\label{version-2}

Para esta versión se sigue manejando una metodologia de lagrange y la
forma de obtener los puntos sigue siendo similar a la anterior versión,
sin embargo para esta se generan más curvas de las necesarias y la
mayoría de estas curvas ya tienen una amplitud negativa, es por esto que
tienen la dirección deseada.

\begin{Shaded}
\begin{Highlighting}[]
\KeywordTok{rm}\NormalTok{(}\DataTypeTok{list=}\KeywordTok{ls}\NormalTok{())}

\NormalTok{graf <-}\StringTok{ }\ControlFlowTok{function}\NormalTok{(arr1,arr2, color)}
\NormalTok{\{}
  \KeywordTok{points}\NormalTok{(arr1,arr2, }\DataTypeTok{pch=}\DecValTok{7}\NormalTok{, }\DataTypeTok{cex=}\FloatTok{0.5}\NormalTok{, }\DataTypeTok{col =}\NormalTok{ color, }\DataTypeTok{asp=}\DecValTok{1}\NormalTok{,}\DataTypeTok{xlab=}\StringTok{"X"}\NormalTok{, }\DataTypeTok{ylab=}\StringTok{"Y"}\NormalTok{, }\DataTypeTok{main=}\StringTok{"Diagrama "}\NormalTok{)}
\NormalTok{\}}

\NormalTok{lagrange<-}\StringTok{ }\ControlFlowTok{function}\NormalTok{(eval, lis, lisy )}
\NormalTok{\{}
\NormalTok{  ret1 <-}\StringTok{ }\NormalTok{( eval }\OperatorTok{-}\StringTok{ }\NormalTok{lis[}\DecValTok{2}\NormalTok{] ) }\OperatorTok{/}\StringTok{ }\NormalTok{( lis[}\DecValTok{1}\NormalTok{] }\OperatorTok{-}\StringTok{ }\NormalTok{lis[}\DecValTok{2}\NormalTok{] ) }\OperatorTok{*}\StringTok{ }\NormalTok{( eval }\OperatorTok{-}\StringTok{ }\NormalTok{lis[}\DecValTok{3}\NormalTok{] ) }\OperatorTok{/}\StringTok{ }\NormalTok{( lis[}\DecValTok{1}\NormalTok{] }\OperatorTok{-}\StringTok{ }\NormalTok{lis[}\DecValTok{3}\NormalTok{] ) }\OperatorTok{*}\StringTok{ }\NormalTok{lisy[}\DecValTok{1}\NormalTok{]}
  
\NormalTok{  ret2 <-}\StringTok{ }\NormalTok{( eval }\OperatorTok{-}\StringTok{ }\NormalTok{lis[}\DecValTok{1}\NormalTok{] ) }\OperatorTok{/}\StringTok{ }\NormalTok{( lis[}\DecValTok{2}\NormalTok{] }\OperatorTok{-}\StringTok{ }\NormalTok{lis[}\DecValTok{1}\NormalTok{] ) }\OperatorTok{*}\StringTok{ }\NormalTok{( eval }\OperatorTok{-}\StringTok{ }\NormalTok{lis[}\DecValTok{3}\NormalTok{] ) }\OperatorTok{/}\StringTok{ }\NormalTok{( lis[}\DecValTok{2}\NormalTok{] }\OperatorTok{-}\StringTok{ }\NormalTok{lis[}\DecValTok{3}\NormalTok{] ) }\OperatorTok{*}\StringTok{ }\NormalTok{lisy[}\DecValTok{2}\NormalTok{]}
  
\NormalTok{  ret3 <-}\StringTok{ }\NormalTok{( eval }\OperatorTok{-}\StringTok{ }\NormalTok{lis[}\DecValTok{1}\NormalTok{] ) }\OperatorTok{/}\StringTok{ }\NormalTok{( lis[}\DecValTok{3}\NormalTok{] }\OperatorTok{-}\StringTok{ }\NormalTok{lis[}\DecValTok{1}\NormalTok{] ) }\OperatorTok{*}\StringTok{ }\NormalTok{( eval }\OperatorTok{-}\StringTok{ }\NormalTok{lis[}\DecValTok{2}\NormalTok{] ) }\OperatorTok{/}\StringTok{ }\NormalTok{( lis[}\DecValTok{3}\NormalTok{] }\OperatorTok{-}\StringTok{ }\NormalTok{lis[}\DecValTok{2}\NormalTok{] ) }\OperatorTok{*}\StringTok{ }\NormalTok{lisy[}\DecValTok{3}\NormalTok{]}
  
  \KeywordTok{return}\NormalTok{ (ret1 }\OperatorTok{+}\StringTok{ }\NormalTok{ret2 }\OperatorTok{+}\StringTok{ }\NormalTok{ret3)}
\NormalTok{\}}

\NormalTok{lag<-}\StringTok{ }\ControlFlowTok{function}\NormalTok{(eval, lis, lisy )}
\NormalTok{\{ }
\NormalTok{  tot <-}\StringTok{ }\DecValTok{0}
  \ControlFlowTok{for}\NormalTok{( j }\ControlFlowTok{in} \DecValTok{1}\OperatorTok{:}\NormalTok{(}\KeywordTok{length}\NormalTok{(lis) ) )}
\NormalTok{  \{}
\NormalTok{    val <-}\StringTok{ }\NormalTok{lisy[j]}
    \ControlFlowTok{for}\NormalTok{( i }\ControlFlowTok{in} \DecValTok{1}\OperatorTok{:}\KeywordTok{length}\NormalTok{(lis)  )}
\NormalTok{    \{}
      \ControlFlowTok{if}\NormalTok{ ( i }\OperatorTok{!=}\StringTok{ }\NormalTok{j)}
\NormalTok{      \{}
\NormalTok{        nn <-}\StringTok{ }\NormalTok{( eval }\OperatorTok{-}\StringTok{ }\NormalTok{lis[i] ) }\OperatorTok{/}\StringTok{ }\NormalTok{( lis[j] }\OperatorTok{-}\StringTok{ }\NormalTok{lis[i] )}
\NormalTok{        val <-}\StringTok{ }\NormalTok{val }\OperatorTok{*}\StringTok{ }\NormalTok{nn}
\NormalTok{      \}}
\NormalTok{    \}}
\NormalTok{    tot <-}\StringTok{ }\NormalTok{tot }\OperatorTok{+}\StringTok{ }\NormalTok{val}
\NormalTok{  \}}
  \KeywordTok{return}\NormalTok{(tot)}
\NormalTok{\}}

\CommentTok{# esto, dada una lista, encuentra los 3 mejores puntos para lagrange, tomando los dos extremos y alguno intermedio}
\NormalTok{findMin <-}\StringTok{ }\ControlFlowTok{function}\NormalTok{(lisx, lisy, liminf, limsup)}
\NormalTok{\{}
  \KeywordTok{print}\NormalTok{(}\StringTok{"------------------------------------------------------"}\NormalTok{)}
  \KeywordTok{print}\NormalTok{(lisx)}
  \KeywordTok{cat}\NormalTok{(liminf,}\StringTok{"  "}\NormalTok{,limsup,}\StringTok{"}\CharTok{\textbackslash{}n}\StringTok{"}\NormalTok{)}
  \KeywordTok{print}\NormalTok{(lisx[liminf}\OperatorTok{:}\NormalTok{limsup])}
\NormalTok{  errmin <-}\StringTok{ }\DecValTok{0}
\NormalTok{  pt <-}\StringTok{ }\DecValTok{0}
\NormalTok{  prim <-}\StringTok{ }\OtherTok{TRUE}
  
\NormalTok{  metodo <-}\StringTok{ "natural"}
  \ControlFlowTok{for}\NormalTok{( i }\ControlFlowTok{in}\NormalTok{ liminf}\OperatorTok{:}\NormalTok{limsup) ## se evitan los dos extremos}
\NormalTok{  \{}
    \KeywordTok{cat}\NormalTok{(}\StringTok{"-----------------------}\CharTok{\textbackslash{}n}\StringTok{"}\NormalTok{)}
    
\NormalTok{    valx <-}\StringTok{ }\NormalTok{lisx[liminf]}
\NormalTok{    valx <-}\StringTok{ }\KeywordTok{c}\NormalTok{(valx, lisx[i])}
\NormalTok{    valx <-}\StringTok{ }\KeywordTok{c}\NormalTok{(valx, lisx[limsup] )}
    
\NormalTok{    valy <-}\StringTok{ }\NormalTok{lisy[liminf]}
\NormalTok{    valy <-}\StringTok{ }\KeywordTok{c}\NormalTok{(valy,lisy[i])}
\NormalTok{    valy <-}\StringTok{ }\KeywordTok{c}\NormalTok{(valy, lisy[limsup] )}
    
\NormalTok{    nx =}\StringTok{ }\KeywordTok{c}\NormalTok{( lisx[liminf], lisx[pt] , lisx[limsup]  )}
\NormalTok{    ny =}\StringTok{ }\KeywordTok{c}\NormalTok{( lisy[liminf], lisy[pt] , lisy[limsup]  )}
    
\NormalTok{    sumerr <-}\StringTok{ }\DecValTok{0}    
    
\NormalTok{    ttam <-}\StringTok{ }\NormalTok{limsup }\OperatorTok{-}\StringTok{ }\NormalTok{liminf}
    \KeywordTok{print}\NormalTok{(valx)}
    \KeywordTok{print}\NormalTok{(valy)}
\NormalTok{    ret <-}\StringTok{ }\NormalTok{(}\KeywordTok{spline}\NormalTok{(valx, valy, }\DataTypeTok{n =}\NormalTok{ ttam ,}\DataTypeTok{method =}\NormalTok{ metodo) )}
    
    \KeywordTok{lines}\NormalTok{(}\KeywordTok{spline}\NormalTok{(valx, valy, }\DataTypeTok{n =}\NormalTok{ ttam, }\DataTypeTok{method =}\NormalTok{ metodo), }\DataTypeTok{col =} \DecValTok{3}\NormalTok{)}
    
\NormalTok{    rx <-}\StringTok{ }\NormalTok{ret}\OperatorTok{$}\NormalTok{x}
\NormalTok{    ry <-}\StringTok{ }\NormalTok{ret}\OperatorTok{$}\NormalTok{y}
    
    \ControlFlowTok{for}\NormalTok{( j }\ControlFlowTok{in} \DecValTok{1}\OperatorTok{:}\NormalTok{ttam )}
\NormalTok{    \{}
\NormalTok{      sumerr <-}\StringTok{ }\NormalTok{sumerr }\OperatorTok{+}\StringTok{ }\KeywordTok{abs}\NormalTok{( lisy[liminf }\OperatorTok{+}\StringTok{ }\NormalTok{j] }\OperatorTok{-}\StringTok{  }\NormalTok{ry[j] ) }
\NormalTok{    \}}
    \ControlFlowTok{if}\NormalTok{ ( }\OperatorTok{!}\KeywordTok{is.nan}\NormalTok{(sumerr) }\OperatorTok{&&}\StringTok{ }\NormalTok{( prim }\OperatorTok{||}\StringTok{ }\NormalTok{sumerr }\OperatorTok{<}\StringTok{ }\NormalTok{errmin ) )}
\NormalTok{    \{}
\NormalTok{      errmin =}\StringTok{ }\NormalTok{sumerr}
\NormalTok{      sumerr =}\StringTok{ }\DecValTok{2}
\NormalTok{      pt =}\StringTok{ }\NormalTok{i }
\NormalTok{      prim =}\StringTok{ }\OtherTok{FALSE}
\NormalTok{    \}}
\NormalTok{  \}}
  \KeywordTok{cat}\NormalTok{ (}\StringTok{"final : "}\NormalTok{,errmin ,}\StringTok{"    "}\NormalTok{,pt,}\StringTok{"}\CharTok{\textbackslash{}n}\StringTok{"}\NormalTok{)}
  
\NormalTok{  nx =}\StringTok{ }\KeywordTok{c}\NormalTok{( lisx[liminf], lisx[pt] , lisx[limsup]  )}
\NormalTok{  ny =}\StringTok{ }\KeywordTok{c}\NormalTok{( lisy[liminf], lisy[pt] , lisy[limsup]  )}
  \KeywordTok{lines}\NormalTok{(}\KeywordTok{spline}\NormalTok{(nx, ny, }\DataTypeTok{n =} \DecValTok{201}\NormalTok{,}\DataTypeTok{method =}\NormalTok{ metodo), }\DataTypeTok{col =} \DecValTok{2}\NormalTok{)}
\NormalTok{\}}

\CommentTok{# reorganizacion de datos }
\NormalTok{x =}\StringTok{ }\KeywordTok{c}\NormalTok{(}\FloatTok{14.65}\NormalTok{, }\FloatTok{14.71}\NormalTok{, }\FloatTok{14.6}\NormalTok{, }\FloatTok{14.8}\NormalTok{, }\FloatTok{15.2}\NormalTok{, }\FloatTok{15.6}\NormalTok{, }\FloatTok{15.7}\NormalTok{, }\FloatTok{17.0}\NormalTok{, }\FloatTok{17.6}\NormalTok{, }\FloatTok{17.52}\NormalTok{, }\FloatTok{17.3}\NormalTok{, }
  \FloatTok{16.8}\NormalTok{, }\FloatTok{15.4}\NormalTok{, }\FloatTok{14.83}\NormalTok{, }\FloatTok{14.4}\NormalTok{, }\FloatTok{14.5}\NormalTok{, }
  \FloatTok{15.0}\NormalTok{, }\FloatTok{15.1}\NormalTok{, }\FloatTok{15.0}\NormalTok{, }\FloatTok{14.9}\NormalTok{, }\FloatTok{14.6}\NormalTok{, }\FloatTok{14.3}\NormalTok{, }\FloatTok{14.0}\NormalTok{, }\FloatTok{13.9}\NormalTok{, }\FloatTok{13.8}\NormalTok{, }\FloatTok{13.5}\NormalTok{, }\FloatTok{13.1}\NormalTok{, }\FloatTok{13.0}\NormalTok{, }
  \FloatTok{13.3}\NormalTok{, }\FloatTok{13.2}\NormalTok{, }\FloatTok{13.1}\NormalTok{, }\FloatTok{12.9}\NormalTok{, }\FloatTok{12.4}\NormalTok{, }\FloatTok{11.9}\NormalTok{, }\FloatTok{11.7}\NormalTok{, }\FloatTok{11.6}\NormalTok{, }\FloatTok{11.3}\NormalTok{, }\FloatTok{10.9}\NormalTok{, }
  \FloatTok{10.7}\NormalTok{, }\FloatTok{10.6}\NormalTok{, }\FloatTok{10.6}\NormalTok{, }\FloatTok{10.1}\NormalTok{, }\FloatTok{9.7}\NormalTok{, }\FloatTok{9.4}\NormalTok{, }\FloatTok{9.3}\NormalTok{, }\FloatTok{9.6}\NormalTok{, }\FloatTok{9.9}\NormalTok{, }\FloatTok{10.1}\NormalTok{, }\FloatTok{10.2}\NormalTok{, }\FloatTok{10.3}\NormalTok{,  }\FloatTok{10.0}\NormalTok{, }\FloatTok{9.5}\NormalTok{, }
  \FloatTok{8.6}\NormalTok{, }\FloatTok{7.5}\NormalTok{, }\FloatTok{7.0}\NormalTok{, }\FloatTok{6.7}\NormalTok{, }\FloatTok{6.6}\NormalTok{, }\FloatTok{7.7}\NormalTok{, }
  \FloatTok{8.0}\NormalTok{, }\FloatTok{8.1}\NormalTok{, }\FloatTok{8.4}\NormalTok{, }\FloatTok{9.2}\NormalTok{, }\FloatTok{9.3}\NormalTok{, }\DecValTok{10}\NormalTok{, }\FloatTok{10.2}\NormalTok{, }\FloatTok{10.3}\NormalTok{)}

\NormalTok{y =}\StringTok{ }\KeywordTok{c}\NormalTok{(}\FloatTok{14.7}\NormalTok{, }\FloatTok{14.33}\NormalTok{, }\FloatTok{13.4}\NormalTok{, }\FloatTok{12.33}\NormalTok{, }\FloatTok{11.0}\NormalTok{, }\FloatTok{10.5}\NormalTok{, }\FloatTok{10.22}\NormalTok{, }\FloatTok{8.2}\NormalTok{, }\FloatTok{7.1}\NormalTok{, }\FloatTok{6.7}\NormalTok{, }\FloatTok{6.6}\NormalTok{, }
  \FloatTok{6.8}\NormalTok{, }\FloatTok{8.3}\NormalTok{, }\FloatTok{8.8}\NormalTok{, }\FloatTok{9.3}\NormalTok{, }\FloatTok{8.8}\NormalTok{, }
  \FloatTok{6.3}\NormalTok{, }\FloatTok{5.5}\NormalTok{, }\FloatTok{5.0}\NormalTok{, }\FloatTok{4.7}\NormalTok{, }\FloatTok{4.6}\NormalTok{, }\FloatTok{4.5}\NormalTok{, }\FloatTok{4.9}\NormalTok{, }\FloatTok{5.4}\NormalTok{, }\FloatTok{5.8}\NormalTok{, }\FloatTok{6.9}\NormalTok{, }\FloatTok{8.2}\NormalTok{, }\FloatTok{7.6}\NormalTok{, }
  \FloatTok{5.8}\NormalTok{, }\FloatTok{4.5}\NormalTok{, }\FloatTok{4.3}\NormalTok{, }\FloatTok{3.9}\NormalTok{, }\FloatTok{4.2}\NormalTok{, }\FloatTok{5.7}\NormalTok{, }\FloatTok{7.0}\NormalTok{, }\FloatTok{7.9}\NormalTok{, }\FloatTok{8.2}\NormalTok{, }\FloatTok{7.3}\NormalTok{, }
  \FloatTok{6.7}\NormalTok{, }\FloatTok{5.5}\NormalTok{, }\FloatTok{5.1}\NormalTok{, }\FloatTok{4.6}\NormalTok{, }\FloatTok{4.7}\NormalTok{, }\FloatTok{5.0}\NormalTok{, }\FloatTok{5.5}\NormalTok{, }\FloatTok{7.2}\NormalTok{, }\FloatTok{7.8}\NormalTok{, }\FloatTok{8.6}\NormalTok{, }\FloatTok{9.4}\NormalTok{, }\FloatTok{10.0}\NormalTok{,  }\FloatTok{10.7}\NormalTok{, }\FloatTok{11.0}\NormalTok{, }
  \FloatTok{9.9}\NormalTok{, }\FloatTok{9.0}\NormalTok{, }\FloatTok{9.1}\NormalTok{, }\FloatTok{9.3}\NormalTok{, }\FloatTok{9.7}\NormalTok{, }\FloatTok{11.7}\NormalTok{, }
  \FloatTok{12.3}\NormalTok{, }\FloatTok{12.5}\NormalTok{, }\FloatTok{13.0}\NormalTok{, }\FloatTok{13.91}\NormalTok{, }\FloatTok{14.9}\NormalTok{, }\DecValTok{16}\NormalTok{, }\FloatTok{16.4}\NormalTok{, }\FloatTok{16.8}\NormalTok{)}


\KeywordTok{plot}\NormalTok{(x,y,}\DataTypeTok{type=}\StringTok{"l"}\NormalTok{,}\DataTypeTok{asp=}\DecValTok{1}\NormalTok{)}
\NormalTok{## de cada uno de estos cortes deberian terminar quedando unos 3 puntos }
\NormalTok{c1 =}\StringTok{ }\DecValTok{9}
\NormalTok{c2 =}\StringTok{ }\DecValTok{15}
\NormalTok{c3 =}\StringTok{ }\DecValTok{27}
\NormalTok{c4 =}\StringTok{ }\DecValTok{37}
\NormalTok{c5 =}\StringTok{ }\DecValTok{52}
\NormalTok{c6 =}\StringTok{ }\DecValTok{56}
\NormalTok{c7 =}\StringTok{ }\KeywordTok{length}\NormalTok{(x)}

\KeywordTok{print}\NormalTok{(}\StringTok{"1"}\NormalTok{)}
\end{Highlighting}
\end{Shaded}

\begin{verbatim}
## [1] "1"
\end{verbatim}

\begin{Shaded}
\begin{Highlighting}[]
\KeywordTok{findMin}\NormalTok{(x,y,}\DecValTok{1}\NormalTok{,c1)}
\end{Highlighting}
\end{Shaded}

\begin{verbatim}
## [1] "------------------------------------------------------"
##  [1] 14.65 14.71 14.60 14.80 15.20 15.60 15.70 17.00 17.60 17.52 17.30
## [12] 16.80 15.40 14.83 14.40 14.50 15.00 15.10 15.00 14.90 14.60 14.30
## [23] 14.00 13.90 13.80 13.50 13.10 13.00 13.30 13.20 13.10 12.90 12.40
## [34] 11.90 11.70 11.60 11.30 10.90 10.70 10.60 10.60 10.10  9.70  9.40
## [45]  9.30  9.60  9.90 10.10 10.20 10.30 10.00  9.50  8.60  7.50  7.00
## [56]  6.70  6.60  7.70  8.00  8.10  8.40  9.20  9.30 10.00 10.20 10.30
## 1    9 
## [1] 14.65 14.71 14.60 14.80 15.20 15.60 15.70 17.00 17.60
## -----------------------
## [1] 14.65 14.65 17.60
## [1] 14.7 14.7  7.1
## -----------------------
## [1] 14.65 14.71 17.60
## [1] 14.70 14.33  7.10
## -----------------------
## [1] 14.65 14.60 17.60
## [1] 14.7 13.4  7.1
## -----------------------
## [1] 14.65 14.80 17.60
## [1] 14.70 12.33  7.10
## -----------------------
## [1] 14.65 15.20 17.60
## [1] 14.7 11.0  7.1
## -----------------------
## [1] 14.65 15.60 17.60
## [1] 14.7 10.5  7.1
## -----------------------
## [1] 14.65 15.70 17.60
## [1] 14.70 10.22  7.10
## -----------------------
## [1] 14.65 17.00 17.60
## [1] 14.7  8.2  7.1
## -----------------------
## [1] 14.65 17.60 17.60
## [1] 14.7  7.1  7.1
## final :  2.331429      1
\end{verbatim}

\begin{Shaded}
\begin{Highlighting}[]
\KeywordTok{print}\NormalTok{(}\StringTok{"2"}\NormalTok{)}
\end{Highlighting}
\end{Shaded}

\begin{verbatim}
## [1] "2"
\end{verbatim}

\begin{Shaded}
\begin{Highlighting}[]
\NormalTok{(}\KeywordTok{findMin}\NormalTok{(x,y,c1,c2) )}
\end{Highlighting}
\end{Shaded}

\begin{verbatim}
## [1] "------------------------------------------------------"
##  [1] 14.65 14.71 14.60 14.80 15.20 15.60 15.70 17.00 17.60 17.52 17.30
## [12] 16.80 15.40 14.83 14.40 14.50 15.00 15.10 15.00 14.90 14.60 14.30
## [23] 14.00 13.90 13.80 13.50 13.10 13.00 13.30 13.20 13.10 12.90 12.40
## [34] 11.90 11.70 11.60 11.30 10.90 10.70 10.60 10.60 10.10  9.70  9.40
## [45]  9.30  9.60  9.90 10.10 10.20 10.30 10.00  9.50  8.60  7.50  7.00
## [56]  6.70  6.60  7.70  8.00  8.10  8.40  9.20  9.30 10.00 10.20 10.30
## 9    15 
## [1] 17.60 17.52 17.30 16.80 15.40 14.83 14.40
## -----------------------
## [1] 17.6 17.6 14.4
## [1] 7.1 7.1 9.3
## -----------------------
## [1] 17.60 17.52 14.40
## [1] 7.1 6.7 9.3
## -----------------------
## [1] 17.6 17.3 14.4
## [1] 7.1 6.6 9.3
## -----------------------
## [1] 17.6 16.8 14.4
## [1] 7.1 6.8 9.3
## -----------------------
## [1] 17.6 15.4 14.4
## [1] 7.1 8.3 9.3
## -----------------------
## [1] 17.60 14.83 14.40
## [1] 7.1 8.8 9.3
## -----------------------
## [1] 17.6 14.4 14.4
## [1] 7.1 9.3 9.3
## final :  10.12852      14
\end{verbatim}

\begin{verbatim}
## NULL
\end{verbatim}

\begin{Shaded}
\begin{Highlighting}[]
\KeywordTok{print}\NormalTok{(}\StringTok{"3"}\NormalTok{)}
\end{Highlighting}
\end{Shaded}

\begin{verbatim}
## [1] "3"
\end{verbatim}

\begin{Shaded}
\begin{Highlighting}[]
\NormalTok{(}\KeywordTok{findMin}\NormalTok{(x,y,c2,c3) )}
\end{Highlighting}
\end{Shaded}

\begin{verbatim}
## [1] "------------------------------------------------------"
##  [1] 14.65 14.71 14.60 14.80 15.20 15.60 15.70 17.00 17.60 17.52 17.30
## [12] 16.80 15.40 14.83 14.40 14.50 15.00 15.10 15.00 14.90 14.60 14.30
## [23] 14.00 13.90 13.80 13.50 13.10 13.00 13.30 13.20 13.10 12.90 12.40
## [34] 11.90 11.70 11.60 11.30 10.90 10.70 10.60 10.60 10.10  9.70  9.40
## [45]  9.30  9.60  9.90 10.10 10.20 10.30 10.00  9.50  8.60  7.50  7.00
## [56]  6.70  6.60  7.70  8.00  8.10  8.40  9.20  9.30 10.00 10.20 10.30
## 15    27 
##  [1] 14.4 14.5 15.0 15.1 15.0 14.9 14.6 14.3 14.0 13.9 13.8 13.5 13.1
## -----------------------
## [1] 14.4 14.4 13.1
## [1] 9.3 9.3 8.2
## -----------------------
## [1] 14.4 14.5 13.1
## [1] 9.3 8.8 8.2
## -----------------------
## [1] 14.4 15.0 13.1
## [1] 9.3 6.3 8.2
## -----------------------
## [1] 14.4 15.1 13.1
## [1] 9.3 5.5 8.2
## -----------------------
## [1] 14.4 15.0 13.1
## [1] 9.3 5.0 8.2
## -----------------------
## [1] 14.4 14.9 13.1
## [1] 9.3 4.7 8.2
## -----------------------
## [1] 14.4 14.6 13.1
## [1] 9.3 4.6 8.2
## -----------------------
## [1] 14.4 14.3 13.1
## [1] 9.3 4.5 8.2
## -----------------------
## [1] 14.4 14.0 13.1
## [1] 9.3 4.9 8.2
## -----------------------
## [1] 14.4 13.9 13.1
## [1] 9.3 5.4 8.2
## -----------------------
## [1] 14.4 13.8 13.1
## [1] 9.3 5.8 8.2
## -----------------------
## [1] 14.4 13.5 13.1
## [1] 9.3 6.9 8.2
## -----------------------
## [1] 14.4 13.1 13.1
## [1] 9.3 8.2 8.2
## final :  6.205663      23
\end{verbatim}

\begin{verbatim}
## NULL
\end{verbatim}

\begin{Shaded}
\begin{Highlighting}[]
\KeywordTok{print}\NormalTok{(}\StringTok{"4"}\NormalTok{)}
\end{Highlighting}
\end{Shaded}

\begin{verbatim}
## [1] "4"
\end{verbatim}

\begin{Shaded}
\begin{Highlighting}[]
\NormalTok{(}\KeywordTok{findMin}\NormalTok{(x,y,c3,c4) ) }
\end{Highlighting}
\end{Shaded}

\begin{verbatim}
## [1] "------------------------------------------------------"
##  [1] 14.65 14.71 14.60 14.80 15.20 15.60 15.70 17.00 17.60 17.52 17.30
## [12] 16.80 15.40 14.83 14.40 14.50 15.00 15.10 15.00 14.90 14.60 14.30
## [23] 14.00 13.90 13.80 13.50 13.10 13.00 13.30 13.20 13.10 12.90 12.40
## [34] 11.90 11.70 11.60 11.30 10.90 10.70 10.60 10.60 10.10  9.70  9.40
## [45]  9.30  9.60  9.90 10.10 10.20 10.30 10.00  9.50  8.60  7.50  7.00
## [56]  6.70  6.60  7.70  8.00  8.10  8.40  9.20  9.30 10.00 10.20 10.30
## 27    37 
##  [1] 13.1 13.0 13.3 13.2 13.1 12.9 12.4 11.9 11.7 11.6 11.3
## -----------------------
## [1] 13.1 13.1 11.3
## [1] 8.2 8.2 8.2
## -----------------------
## [1] 13.1 13.0 11.3
## [1] 8.2 7.6 8.2
## -----------------------
## [1] 13.1 13.3 11.3
## [1] 8.2 5.8 8.2
## -----------------------
## [1] 13.1 13.2 11.3
## [1] 8.2 4.5 8.2
## -----------------------
## [1] 13.1 13.1 11.3
## [1] 8.2 4.3 8.2
## -----------------------
## [1] 13.1 12.9 11.3
## [1] 8.2 3.9 8.2
## -----------------------
## [1] 13.1 12.4 11.3
## [1] 8.2 4.2 8.2
## -----------------------
## [1] 13.1 11.9 11.3
## [1] 8.2 5.7 8.2
## -----------------------
## [1] 13.1 11.7 11.3
## [1] 8.2 7.0 8.2
## -----------------------
## [1] 13.1 11.6 11.3
## [1] 8.2 7.9 8.2
## -----------------------
## [1] 13.1 11.3 11.3
## [1] 8.2 8.2 8.2
## final :  7.904554      33
\end{verbatim}

\begin{verbatim}
## NULL
\end{verbatim}

\begin{Shaded}
\begin{Highlighting}[]
\KeywordTok{print}\NormalTok{(}\StringTok{"5"}\NormalTok{)}
\end{Highlighting}
\end{Shaded}

\begin{verbatim}
## [1] "5"
\end{verbatim}

\begin{Shaded}
\begin{Highlighting}[]
\NormalTok{(}\KeywordTok{findMin}\NormalTok{(x,y,c4,c5) )}
\end{Highlighting}
\end{Shaded}

\begin{verbatim}
## [1] "------------------------------------------------------"
##  [1] 14.65 14.71 14.60 14.80 15.20 15.60 15.70 17.00 17.60 17.52 17.30
## [12] 16.80 15.40 14.83 14.40 14.50 15.00 15.10 15.00 14.90 14.60 14.30
## [23] 14.00 13.90 13.80 13.50 13.10 13.00 13.30 13.20 13.10 12.90 12.40
## [34] 11.90 11.70 11.60 11.30 10.90 10.70 10.60 10.60 10.10  9.70  9.40
## [45]  9.30  9.60  9.90 10.10 10.20 10.30 10.00  9.50  8.60  7.50  7.00
## [56]  6.70  6.60  7.70  8.00  8.10  8.40  9.20  9.30 10.00 10.20 10.30
## 37    52 
##  [1] 11.3 10.9 10.7 10.6 10.6 10.1  9.7  9.4  9.3  9.6  9.9 10.1 10.2 10.3
## [15] 10.0  9.5
## -----------------------
## [1] 11.3 11.3  9.5
## [1]  8.2  8.2 11.0
## -----------------------
## [1] 11.3 10.9  9.5
## [1]  8.2  7.3 11.0
## -----------------------
## [1] 11.3 10.7  9.5
## [1]  8.2  6.7 11.0
## -----------------------
## [1] 11.3 10.6  9.5
## [1]  8.2  5.5 11.0
## -----------------------
## [1] 11.3 10.6  9.5
## [1]  8.2  5.1 11.0
## -----------------------
## [1] 11.3 10.1  9.5
## [1]  8.2  4.6 11.0
## -----------------------
## [1] 11.3  9.7  9.5
## [1]  8.2  4.7 11.0
## -----------------------
## [1] 11.3  9.4  9.5
## [1]  8.2  5.0 11.0
## -----------------------
## [1] 11.3  9.3  9.5
## [1]  8.2  5.5 11.0
## -----------------------
## [1] 11.3  9.6  9.5
## [1]  8.2  7.2 11.0
## -----------------------
## [1] 11.3  9.9  9.5
## [1]  8.2  7.8 11.0
## -----------------------
## [1] 11.3 10.1  9.5
## [1]  8.2  8.6 11.0
## -----------------------
## [1] 11.3 10.2  9.5
## [1]  8.2  9.4 11.0
## -----------------------
## [1] 11.3 10.3  9.5
## [1]  8.2 10.0 11.0
## -----------------------
## [1] 11.3 10.0  9.5
## [1]  8.2 10.7 11.0
## -----------------------
## [1] 11.3  9.5  9.5
## [1]  8.2 11.0 11.0
## final :  34.88317      47
\end{verbatim}

\begin{verbatim}
## NULL
\end{verbatim}

\begin{Shaded}
\begin{Highlighting}[]
\KeywordTok{print}\NormalTok{(}\StringTok{"6"}\NormalTok{)}
\end{Highlighting}
\end{Shaded}

\begin{verbatim}
## [1] "6"
\end{verbatim}

\begin{Shaded}
\begin{Highlighting}[]
\NormalTok{(}\KeywordTok{findMin}\NormalTok{(x,y,c5,c6) )}
\end{Highlighting}
\end{Shaded}

\begin{verbatim}
## [1] "------------------------------------------------------"
##  [1] 14.65 14.71 14.60 14.80 15.20 15.60 15.70 17.00 17.60 17.52 17.30
## [12] 16.80 15.40 14.83 14.40 14.50 15.00 15.10 15.00 14.90 14.60 14.30
## [23] 14.00 13.90 13.80 13.50 13.10 13.00 13.30 13.20 13.10 12.90 12.40
## [34] 11.90 11.70 11.60 11.30 10.90 10.70 10.60 10.60 10.10  9.70  9.40
## [45]  9.30  9.60  9.90 10.10 10.20 10.30 10.00  9.50  8.60  7.50  7.00
## [56]  6.70  6.60  7.70  8.00  8.10  8.40  9.20  9.30 10.00 10.20 10.30
## 52    56 
## [1] 9.5 8.6 7.5 7.0 6.7
## -----------------------
## [1] 9.5 9.5 6.7
## [1] 11.0 11.0  9.3
## -----------------------
## [1] 9.5 8.6 6.7
## [1] 11.0  9.9  9.3
## -----------------------
## [1] 9.5 7.5 6.7
## [1] 11.0  9.0  9.3
## -----------------------
## [1] 9.5 7.0 6.7
## [1] 11.0  9.1  9.3
## -----------------------
## [1] 9.5 6.7 6.7
## [1] 11.0  9.3  9.3
## final :  2.923333      54
\end{verbatim}

\begin{verbatim}
## NULL
\end{verbatim}

\begin{Shaded}
\begin{Highlighting}[]
\KeywordTok{print}\NormalTok{(}\StringTok{"7"}\NormalTok{)}
\end{Highlighting}
\end{Shaded}

\begin{verbatim}
## [1] "7"
\end{verbatim}

\begin{Shaded}
\begin{Highlighting}[]
\NormalTok{(}\KeywordTok{findMin}\NormalTok{(x,y,c6,c7) )}
\end{Highlighting}
\end{Shaded}

\includegraphics{Documento_notebook_files/figure-latex/unnamed-chunk-5-1.pdf}

\begin{verbatim}
## [1] "------------------------------------------------------"
##  [1] 14.65 14.71 14.60 14.80 15.20 15.60 15.70 17.00 17.60 17.52 17.30
## [12] 16.80 15.40 14.83 14.40 14.50 15.00 15.10 15.00 14.90 14.60 14.30
## [23] 14.00 13.90 13.80 13.50 13.10 13.00 13.30 13.20 13.10 12.90 12.40
## [34] 11.90 11.70 11.60 11.30 10.90 10.70 10.60 10.60 10.10  9.70  9.40
## [45]  9.30  9.60  9.90 10.10 10.20 10.30 10.00  9.50  8.60  7.50  7.00
## [56]  6.70  6.60  7.70  8.00  8.10  8.40  9.20  9.30 10.00 10.20 10.30
## 56    66 
##  [1]  6.7  6.6  7.7  8.0  8.1  8.4  9.2  9.3 10.0 10.2 10.3
## -----------------------
## [1]  6.7  6.7 10.3
## [1]  9.3  9.3 16.8
## -----------------------
## [1]  6.7  6.6 10.3
## [1]  9.3  9.7 16.8
## -----------------------
## [1]  6.7  7.7 10.3
## [1]  9.3 11.7 16.8
## -----------------------
## [1]  6.7  8.0 10.3
## [1]  9.3 12.3 16.8
## -----------------------
## [1]  6.7  8.1 10.3
## [1]  9.3 12.5 16.8
## -----------------------
## [1]  6.7  8.4 10.3
## [1]  9.3 13.0 16.8
## -----------------------
## [1]  6.7  9.2 10.3
## [1]  9.30 13.91 16.80
## -----------------------
## [1]  6.7  9.3 10.3
## [1]  9.3 14.9 16.8
## -----------------------
## [1]  6.7 10.0 10.3
## [1]  9.3 16.0 16.8
## -----------------------
## [1]  6.7 10.2 10.3
## [1]  9.3 16.4 16.8
## -----------------------
## [1]  6.7 10.3 10.3
## [1]  9.3 16.8 16.8
## final :  4.612414      58
\end{verbatim}

\begin{verbatim}
## NULL
\end{verbatim}

\begin{Shaded}
\begin{Highlighting}[]
\KeywordTok{print}\NormalTok{(}\StringTok{"8"}\NormalTok{)}
\end{Highlighting}
\end{Shaded}

\begin{verbatim}
## [1] "8"
\end{verbatim}

\subsubsection{Versión final}\label{version-final}

Para esta versión se tomó la decisión de utilizar un algoritmo que
tomará un punto del vector original y realizará segmentos hacia
diferentes puntos del vector original, estos segmentos se calcularon
bajo el funcionamiento de interpolación por spline, con base en esos
segmentos calculará el error respecto al segmento original y finalmente
se quedará con el segmento que presenta el error mínimo entre los
segmentos obtenidos; tras obtener un segmento el punto de partida para
la búsqueda de los próximos segmentos es el punto de cierre del segmento
inmediatamente anterior.

Debido a la cantidad de comparaciones que realiza el sistema, el
algoritmo presenta un tiempo relativamente elevado para encontrar la
solucion.

\begin{Shaded}
\begin{Highlighting}[]
\KeywordTok{rm}\NormalTok{(}\DataTypeTok{list=}\KeywordTok{ls}\NormalTok{())}

\NormalTok{metodo <-}\StringTok{ "fmm"}  \CommentTok{# metodos : ("fmm","natural","periodic") }

\NormalTok{graf <-}\StringTok{ }\ControlFlowTok{function}\NormalTok{(arr1,arr2, color)}
\NormalTok{\{}
  \KeywordTok{points}\NormalTok{(arr1,arr2, }\DataTypeTok{pch=}\DecValTok{7}\NormalTok{, }\DataTypeTok{cex=}\FloatTok{0.5}\NormalTok{, }\DataTypeTok{col =}\NormalTok{ color, }\DataTypeTok{asp=}\DecValTok{1}\NormalTok{,}\DataTypeTok{xlab=}\StringTok{"X"}\NormalTok{, }\DataTypeTok{ylab=}\StringTok{"Y"}\NormalTok{, }\DataTypeTok{main=}\StringTok{"Diagrama "}\NormalTok{)}
\NormalTok{\}}

\CommentTok{# esto, dada una lista, encuentra los 3 mejores puntos para lagrange, tomando los dos extremos y alguno intermedio}
\NormalTok{findMin <-}\StringTok{ }\ControlFlowTok{function}\NormalTok{(lisx, lisy, liminf, limsup)}
\NormalTok{\{}
\NormalTok{  errmin <-}\StringTok{ }\DecValTok{0}
\NormalTok{  pt     <-}\StringTok{ }\DecValTok{0}
\NormalTok{  prim   <-}\StringTok{ }\OtherTok{TRUE}
\NormalTok{  intersec  <-}\StringTok{ }\DecValTok{0}
\NormalTok{  bstintrsc <-}\StringTok{ }\DecValTok{0}
  
  \ControlFlowTok{for}\NormalTok{( i }\ControlFlowTok{in}\NormalTok{ liminf}\OperatorTok{:}\NormalTok{limsup)}
\NormalTok{  \{}
\NormalTok{    valx <-}\StringTok{          }\NormalTok{lisx[liminf]}
\NormalTok{    valx <-}\StringTok{ }\KeywordTok{c}\NormalTok{( valx, lisx[i] )}
\NormalTok{    valx <-}\StringTok{ }\KeywordTok{c}\NormalTok{( valx, lisx[limsup] )}
    
\NormalTok{    valy <-}\StringTok{          }\NormalTok{lisy[liminf]}
\NormalTok{    valy <-}\StringTok{ }\KeywordTok{c}\NormalTok{( valy, lisy[i] )}
\NormalTok{    valy <-}\StringTok{ }\KeywordTok{c}\NormalTok{( valy, lisy[limsup] )}
    
\NormalTok{    sumerr <-}\StringTok{ }\DecValTok{0}    
    
\NormalTok{    ttam <-}\StringTok{ }\NormalTok{limsup }\OperatorTok{-}\StringTok{ }\NormalTok{liminf}
\NormalTok{    ret  <-}\StringTok{ }\KeywordTok{spline}\NormalTok{( valx , valy , }\DataTypeTok{n =}\NormalTok{ ttam , }\DataTypeTok{method =}\NormalTok{ metodo)}
\NormalTok{    ry   <-}\StringTok{ }\NormalTok{ret}\OperatorTok{$}\NormalTok{y}
    
\NormalTok{    intersec <-}\StringTok{ }\DecValTok{0}
    
    \CommentTok{#lines(spline(valx, valy, n = ttam, method = metodo), col = 3)}
    
    \ControlFlowTok{for}\NormalTok{( j }\ControlFlowTok{in} \DecValTok{1}\OperatorTok{:}\NormalTok{ttam )}
\NormalTok{    \{}
      \ControlFlowTok{if}\NormalTok{ (lisy[ liminf }\OperatorTok{+}\StringTok{ }\NormalTok{j ] }\OperatorTok{==}\StringTok{  }\NormalTok{ry[ j ])}
\NormalTok{      \{}
\NormalTok{        intersec <-}\StringTok{ }\NormalTok{intersec }\OperatorTok{+}\StringTok{ }\DecValTok{1}
        \CommentTok{#cat (lisy[ liminf + j ]  ,"-----",  ry[ j ],"\textbackslash{}n")}
\NormalTok{      \}}
\NormalTok{      sumerr <-}\StringTok{ }\NormalTok{sumerr }\OperatorTok{+}\StringTok{ }\KeywordTok{abs}\NormalTok{( lisy[ liminf }\OperatorTok{+}\StringTok{ }\NormalTok{j ] }\OperatorTok{-}\StringTok{  }\NormalTok{ry[ j ] ) }
\NormalTok{    \}}
    \ControlFlowTok{if}\NormalTok{ ( }\OperatorTok{!}\KeywordTok{is.nan}\NormalTok{( sumerr ) }\OperatorTok{&&}\StringTok{ }\NormalTok{( prim }\OperatorTok{||}\StringTok{ }\NormalTok{sumerr }\OperatorTok{<}\StringTok{ }\NormalTok{errmin ) )}
\NormalTok{    \{}
\NormalTok{      errmin <-}\StringTok{ }\NormalTok{sumerr}
\NormalTok{      pt     <-}\StringTok{ }\NormalTok{i }
\NormalTok{      prim   <-}\StringTok{ }\OtherTok{FALSE}
\NormalTok{      bstintrsc <-}\StringTok{ }\NormalTok{intersec}
\NormalTok{    \}}
\NormalTok{  \}}
  \CommentTok{#cat("best intersectioon : " , bstintrsc," \textbackslash{}n")}
  \CommentTok{#cat ("final : ",errmin ,"    ",pt,"\textbackslash{}n")}
  \KeywordTok{return}\NormalTok{( }\KeywordTok{c}\NormalTok{(pt,errmin,bstintrsc) )}
\NormalTok{\}}

\NormalTok{puntos <-}\StringTok{ }\ControlFlowTok{function}\NormalTok{(x,y)}
\NormalTok{\{}
\NormalTok{  totintersect <-}\StringTok{ }\DecValTok{0}
\NormalTok{  increm <-}\StringTok{ }\FloatTok{1.4}
\NormalTok{  errorinicial  <-}\StringTok{ }\DecValTok{1}
\NormalTok{  erroraceptado <-}\StringTok{ }\NormalTok{errorinicial}
\NormalTok{  puntos <-}\StringTok{ }\KeywordTok{c}\NormalTok{(}\DecValTok{1}\NormalTok{)}
\NormalTok{  top    <-}\StringTok{ }\DecValTok{1} 
\NormalTok{  min    <-}\StringTok{ }\KeywordTok{c}\NormalTok{(}\OtherTok{Inf}\NormalTok{,}\OtherTok{Inf}\NormalTok{,}\OtherTok{Inf}\NormalTok{,}\OtherTok{Inf}\NormalTok{,}\OtherTok{Inf}\NormalTok{)}\CommentTok{# de, medio , hasta, valor, puntos intersectantes }
\NormalTok{  i      <-}\StringTok{ }\DecValTok{3}
\NormalTok{  prim   <-}\StringTok{ }\OtherTok{TRUE}
\NormalTok{  len    <-}\StringTok{ }\DecValTok{0} \CommentTok{# longitud del tramo actual }
  
  \ControlFlowTok{while}\NormalTok{ ( top }\OperatorTok{!=}\StringTok{ }\KeywordTok{length}\NormalTok{(x) )}
\NormalTok{  \{}
    \ControlFlowTok{if}\NormalTok{ ( i }\OperatorTok{-}\StringTok{ }\NormalTok{top }\OperatorTok{>=}\StringTok{ }\DecValTok{1}\NormalTok{)}
\NormalTok{    \{}
\NormalTok{      ret <-}\StringTok{ }\KeywordTok{findMin}\NormalTok{(x,y,top,i) }
      
      \ControlFlowTok{if}\NormalTok{ ( }\KeywordTok{abs}\NormalTok{(min[}\DecValTok{4}\NormalTok{] }\OperatorTok{-}\StringTok{ }\NormalTok{ret[}\DecValTok{2}\NormalTok{]) }\OperatorTok{<=}\StringTok{ }\NormalTok{erroraceptado }\OperatorTok{||}\StringTok{ }\NormalTok{prim) }\CommentTok{# si se encuentran otros valores que minimicen el error, se guardan. o si es el primer valor revisado}
\NormalTok{      \{}
\NormalTok{        min <-}\StringTok{ }\KeywordTok{c}\NormalTok{( top ,ret[}\DecValTok{1}\NormalTok{], i, ret[}\DecValTok{2}\NormalTok{],ret[}\DecValTok{3}\NormalTok{])}
\NormalTok{        prim =}\StringTok{ }\OtherTok{FALSE}
\NormalTok{        erroraceptado <-}\StringTok{ }\NormalTok{erroraceptado }\OperatorTok{+}\StringTok{ }\NormalTok{increm}\OperatorTok{*}\NormalTok{(len}\OperatorTok{/}\KeywordTok{length}\NormalTok{(x)) }\CommentTok{# para darle mas }
\NormalTok{        len <-}\StringTok{ }\DecValTok{1} \CommentTok{# se reinicia la longitud actual }
\NormalTok{      \}}
      \ControlFlowTok{else}
\NormalTok{      \{}
\NormalTok{        len <-}\StringTok{ }\NormalTok{len }\OperatorTok{+}\StringTok{ }\DecValTok{1} \CommentTok{# aumenta la longitud del tramo actual}
\NormalTok{      \}}
      
      \ControlFlowTok{if}\NormalTok{ ( i }\OperatorTok{>=}\StringTok{ }\KeywordTok{length}\NormalTok{(x) )}
\NormalTok{      \{}
\NormalTok{        nx <-}\StringTok{ }\KeywordTok{c}\NormalTok{( x[ min[}\DecValTok{1}\NormalTok{] ], x[ min[}\DecValTok{2}\NormalTok{] ] , x[ min[}\DecValTok{3}\NormalTok{] ] )}
\NormalTok{        ny <-}\StringTok{ }\KeywordTok{c}\NormalTok{( y[ min[}\DecValTok{1}\NormalTok{] ], y[ min[}\DecValTok{2}\NormalTok{] ] , y[ min[}\DecValTok{3}\NormalTok{] ] )}
        \KeywordTok{lines}\NormalTok{(}\KeywordTok{spline}\NormalTok{(nx, ny, }\DataTypeTok{n =} \DecValTok{200}\NormalTok{, }\DataTypeTok{method =}\NormalTok{ metodo ), }\DataTypeTok{col =} \DecValTok{2}\NormalTok{)}
        
\NormalTok{        erroraceptado <-}\StringTok{ }\NormalTok{errorinicial }\CommentTok{# reiniciar el error aceptado}
        
        \KeywordTok{cat}\NormalTok{(}\StringTok{"desde : "}\NormalTok{,min[}\DecValTok{1}\NormalTok{],}\StringTok{"}\CharTok{\textbackslash{}t}\StringTok{punto medio : "}\NormalTok{,min[}\DecValTok{2}\NormalTok{], }\StringTok{"}\CharTok{\textbackslash{}t}\StringTok{hasta : "}\NormalTok{,min[}\DecValTok{3}\NormalTok{], }\StringTok{"}\CharTok{\textbackslash{}t}\StringTok{puntos intersectantes : "}\NormalTok{,min[}\DecValTok{5}\NormalTok{],}\StringTok{"}\CharTok{\textbackslash{}t}\StringTok{error segmento : "}\NormalTok{,min[}\DecValTok{4}\NormalTok{],}\StringTok{"}\CharTok{\textbackslash{}n}\StringTok{"}\NormalTok{)}
\NormalTok{        totintersect <-}\StringTok{ }\NormalTok{totintersect }\OperatorTok{+}\StringTok{ }\NormalTok{min[}\DecValTok{5}\NormalTok{]}
        \ControlFlowTok{if}\NormalTok{ ( min[}\DecValTok{2}\NormalTok{] }\OperatorTok{==}\StringTok{ }\NormalTok{puntos[ }\KeywordTok{length}\NormalTok{( puntos ) ] ) }
\NormalTok{        \{}
\NormalTok{          puntos <-}\StringTok{ }\KeywordTok{c}\NormalTok{( puntos,min[ }\DecValTok{3}\NormalTok{ ] ) }\CommentTok{# min[1] en teoria es el ultimo de los anteriores, entonces ya esta }
\NormalTok{        \}}
        \ControlFlowTok{else}
\NormalTok{        \{}
\NormalTok{          puntos <-}\StringTok{ }\KeywordTok{c}\NormalTok{( puntos,min[ }\DecValTok{2}\OperatorTok{:}\DecValTok{3}\NormalTok{ ] )}
\NormalTok{        \}}
        
\NormalTok{        top <-}\StringTok{ }\NormalTok{min[}\DecValTok{3}\NormalTok{]}
\NormalTok{        i   <-}\StringTok{ }\NormalTok{top}
        
\NormalTok{        min  <-}\StringTok{ }\KeywordTok{c}\NormalTok{(}\OtherTok{Inf}\NormalTok{,}\OtherTok{Inf}\NormalTok{,}\OtherTok{Inf}\NormalTok{,}\OtherTok{Inf}\NormalTok{,}\OtherTok{Inf}\NormalTok{)}\CommentTok{# de, medio , hasta, valor, interseccion}
\NormalTok{        prim <-}\StringTok{ }\OtherTok{TRUE}
\NormalTok{      \}}
      \ControlFlowTok{else} 
\NormalTok{      \{}
\NormalTok{        i <-}\StringTok{ }\NormalTok{i }\OperatorTok{+}\StringTok{ }\DecValTok{1}  
\NormalTok{      \}}
\NormalTok{    \}}
    \ControlFlowTok{else} 
\NormalTok{    \{}
\NormalTok{      i <-}\StringTok{ }\NormalTok{i }\OperatorTok{+}\StringTok{ }\DecValTok{1}  
\NormalTok{    \}}
    
    \ControlFlowTok{if}\NormalTok{ ( i }\OperatorTok{>=}\StringTok{ }\KeywordTok{length}\NormalTok{(x) ) }\CommentTok{# por si acaso :v }
\NormalTok{    \{}
\NormalTok{      i <-}\StringTok{ }\KeywordTok{length}\NormalTok{(x)}
\NormalTok{    \}}
\NormalTok{  \}}
  \KeywordTok{cat}\NormalTok{(}\StringTok{"}\CharTok{\textbackslash{}n\textbackslash{}n}\StringTok{cantidad : "}\NormalTok{,}\KeywordTok{length}\NormalTok{(puntos),}\StringTok{"}\CharTok{\textbackslash{}t}\StringTok{indice de Jaccard : "}\NormalTok{,totintersect}\OperatorTok{/}\KeywordTok{length}\NormalTok{(x),}\StringTok{"}\CharTok{\textbackslash{}n}\StringTok{puntos seleccionados : "}\NormalTok{,puntos,}\StringTok{"}\CharTok{\textbackslash{}n}\StringTok{"}\NormalTok{)}
\NormalTok{\}}

\CommentTok{# reorganizacion de datos }
\NormalTok{x =}\StringTok{ }\KeywordTok{c}\NormalTok{(}\FloatTok{14.65}\NormalTok{, }\FloatTok{14.71}\NormalTok{, }\FloatTok{14.6}\NormalTok{, }\FloatTok{14.8}\NormalTok{, }\FloatTok{15.2}\NormalTok{, }\FloatTok{15.6}\NormalTok{, }\FloatTok{15.7}\NormalTok{, }\FloatTok{17.0}\NormalTok{, }\FloatTok{17.6}\NormalTok{, }\FloatTok{17.52}\NormalTok{, }\FloatTok{17.3}\NormalTok{, }
  \FloatTok{16.8}\NormalTok{, }\FloatTok{15.4}\NormalTok{, }\FloatTok{14.83}\NormalTok{, }\FloatTok{14.4}\NormalTok{, }\FloatTok{14.5}\NormalTok{, }
  \FloatTok{15.0}\NormalTok{, }\FloatTok{15.1}\NormalTok{, }\FloatTok{15.0}\NormalTok{, }\FloatTok{14.9}\NormalTok{, }\FloatTok{14.6}\NormalTok{, }\FloatTok{14.3}\NormalTok{, }\FloatTok{14.0}\NormalTok{, }\FloatTok{13.9}\NormalTok{, }\FloatTok{13.8}\NormalTok{, }\FloatTok{13.5}\NormalTok{, }\FloatTok{13.1}\NormalTok{, }\FloatTok{13.0}\NormalTok{, }
  \FloatTok{13.3}\NormalTok{, }\FloatTok{13.2}\NormalTok{, }\FloatTok{13.1}\NormalTok{, }\FloatTok{12.9}\NormalTok{, }\FloatTok{12.4}\NormalTok{, }\FloatTok{11.9}\NormalTok{, }\FloatTok{11.7}\NormalTok{, }\FloatTok{11.6}\NormalTok{, }\FloatTok{11.3}\NormalTok{, }\FloatTok{10.9}\NormalTok{, }
  \FloatTok{10.7}\NormalTok{, }\FloatTok{10.6}\NormalTok{, }\FloatTok{10.6}\NormalTok{, }\FloatTok{10.1}\NormalTok{, }\FloatTok{9.7}\NormalTok{, }\FloatTok{9.4}\NormalTok{, }\FloatTok{9.3}\NormalTok{, }\FloatTok{9.6}\NormalTok{, }\FloatTok{9.9}\NormalTok{, }\FloatTok{10.1}\NormalTok{, }\FloatTok{10.2}\NormalTok{, }\FloatTok{10.3}\NormalTok{,  }\FloatTok{10.0}\NormalTok{, }\FloatTok{9.5}\NormalTok{, }
  \FloatTok{8.6}\NormalTok{, }\FloatTok{7.5}\NormalTok{, }\FloatTok{7.0}\NormalTok{, }\FloatTok{6.7}\NormalTok{, }\FloatTok{6.6}\NormalTok{, }\FloatTok{7.7}\NormalTok{, }
  \FloatTok{8.0}\NormalTok{, }\FloatTok{8.1}\NormalTok{, }\FloatTok{8.4}\NormalTok{, }\FloatTok{9.2}\NormalTok{, }\FloatTok{9.3}\NormalTok{, }\DecValTok{10}\NormalTok{, }\FloatTok{10.2}\NormalTok{, }\FloatTok{10.3}\NormalTok{)}

\NormalTok{y =}\StringTok{ }\KeywordTok{c}\NormalTok{(}\FloatTok{14.7}\NormalTok{, }\FloatTok{14.33}\NormalTok{, }\FloatTok{13.4}\NormalTok{, }\FloatTok{12.33}\NormalTok{, }\FloatTok{11.0}\NormalTok{, }\FloatTok{10.5}\NormalTok{, }\FloatTok{10.22}\NormalTok{, }\FloatTok{8.2}\NormalTok{, }\FloatTok{7.1}\NormalTok{, }\FloatTok{6.7}\NormalTok{, }\FloatTok{6.6}\NormalTok{, }
  \FloatTok{6.8}\NormalTok{, }\FloatTok{8.3}\NormalTok{, }\FloatTok{8.8}\NormalTok{, }\FloatTok{9.3}\NormalTok{, }\FloatTok{8.8}\NormalTok{, }
  \FloatTok{6.3}\NormalTok{, }\FloatTok{5.5}\NormalTok{, }\FloatTok{5.0}\NormalTok{, }\FloatTok{4.7}\NormalTok{, }\FloatTok{4.6}\NormalTok{, }\FloatTok{4.5}\NormalTok{, }\FloatTok{4.9}\NormalTok{, }\FloatTok{5.4}\NormalTok{, }\FloatTok{5.8}\NormalTok{, }\FloatTok{6.9}\NormalTok{, }\FloatTok{8.2}\NormalTok{, }\FloatTok{7.6}\NormalTok{, }
  \FloatTok{5.8}\NormalTok{, }\FloatTok{4.5}\NormalTok{, }\FloatTok{4.3}\NormalTok{, }\FloatTok{3.9}\NormalTok{, }\FloatTok{4.2}\NormalTok{, }\FloatTok{5.7}\NormalTok{, }\FloatTok{7.0}\NormalTok{, }\FloatTok{7.9}\NormalTok{, }\FloatTok{8.2}\NormalTok{, }\FloatTok{7.3}\NormalTok{, }
  \FloatTok{6.7}\NormalTok{, }\FloatTok{5.5}\NormalTok{, }\FloatTok{5.1}\NormalTok{, }\FloatTok{4.6}\NormalTok{, }\FloatTok{4.7}\NormalTok{, }\FloatTok{5.0}\NormalTok{, }\FloatTok{5.5}\NormalTok{, }\FloatTok{7.2}\NormalTok{, }\FloatTok{7.8}\NormalTok{, }\FloatTok{8.6}\NormalTok{, }\FloatTok{9.4}\NormalTok{, }\FloatTok{10.0}\NormalTok{,  }\FloatTok{10.7}\NormalTok{, }\FloatTok{11.0}\NormalTok{, }
  \FloatTok{9.9}\NormalTok{, }\FloatTok{9.0}\NormalTok{, }\FloatTok{9.1}\NormalTok{, }\FloatTok{9.3}\NormalTok{, }\FloatTok{9.7}\NormalTok{, }\FloatTok{11.7}\NormalTok{, }
  \FloatTok{12.3}\NormalTok{, }\FloatTok{12.5}\NormalTok{, }\FloatTok{13.0}\NormalTok{, }\FloatTok{13.91}\NormalTok{, }\FloatTok{14.9}\NormalTok{, }\DecValTok{16}\NormalTok{, }\FloatTok{16.4}\NormalTok{, }\FloatTok{16.8}\NormalTok{)}


\KeywordTok{plot}\NormalTok{(x,y,}\DataTypeTok{type=}\StringTok{"l"}\NormalTok{,}\DataTypeTok{asp=}\DecValTok{1}\NormalTok{,}\DataTypeTok{main=}\StringTok{"Mano"}\NormalTok{)}

\KeywordTok{puntos}\NormalTok{(x,y)}
\end{Highlighting}
\end{Shaded}

\includegraphics{Documento_notebook_files/figure-latex/unnamed-chunk-6-1.pdf}

\begin{verbatim}
## desde :  1   punto medio :  1    hasta :  9  puntos intersectantes :  1  error segmento :  2.331429 
## desde :  9   punto medio :  11   hasta :  12     puntos intersectantes :  0  error segmento :  0.4633333 
## desde :  12  punto medio :  12   hasta :  13     puntos intersectantes :  1  error segmento :  0 
## desde :  13  punto medio :  14   hasta :  16     puntos intersectantes :  1  error segmento :  1.052632 
## desde :  16  punto medio :  19   hasta :  20     puntos intersectantes :  0  error segmento :  3.566667 
## desde :  20  punto medio :  21   hasta :  25     puntos intersectantes :  0  error segmento :  3.904167 
## desde :  25  punto medio :  25   hasta :  26     puntos intersectantes :  1  error segmento :  0 
## desde :  26  punto medio :  27   hasta :  29     puntos intersectantes :  1  error segmento :  2.9 
## desde :  29  punto medio :  33   hasta :  34     puntos intersectantes :  0  error segmento :  1.969444 
## desde :  34  punto medio :  37   hasta :  39     puntos intersectantes :  1  error segmento :  1.5 
## desde :  39  punto medio :  39   hasta :  40     puntos intersectantes :  1  error segmento :  0 
## desde :  40  punto medio :  43   hasta :  45     puntos intersectantes :  1  error segmento :  0.9388889 
## desde :  45  punto medio :  46   hasta :  50     puntos intersectantes :  1  error segmento :  3.208333 
## desde :  50  punto medio :  52   hasta :  53     puntos intersectantes :  0  error segmento :  0.9006944 
## desde :  53  punto medio :  54   hasta :  57     puntos intersectantes :  0  error segmento :  1.118631 
## desde :  57  punto medio :  63   hasta :  65     puntos intersectantes :  1  error segmento :  5.243333 
## desde :  65  punto medio :  65   hasta :  66     puntos intersectantes :  0  error segmento :  0.4 
## 
## 
## cantidad :  30   indice de Jaccard :  0.1515152 
## puntos seleccionados :  1 9 11 12 13 14 16 19 20 21 25 26 27 29 33 34 37 39 40 43 45 46 50 52 53 54 57 63 65 66
\end{verbatim}

Sin embargo a pesar de que la versión final se presentarón 30 puntos,
aun se ven algunas anomalias, es por esto que se plantea una variación
al algoritmo presentado anteriormente para que en el caso donde se
empieza a generar el primer lado de la mano se realizara un pequeño giro
donde y pasa a ser x y x pasa a ser y debido a que en ese sector no se
presenta una función y es por eso que se presenta la anomalia. A
continuación se muestra el algoritmo

\begin{Shaded}
\begin{Highlighting}[]
\KeywordTok{rm}\NormalTok{(}\DataTypeTok{list=}\KeywordTok{ls}\NormalTok{())}

\NormalTok{metodo <-}\StringTok{ "fmm"}  \CommentTok{# metodos : ("fmm","natural","periodic") }

\NormalTok{graf <-}\StringTok{ }\ControlFlowTok{function}\NormalTok{(arr1,arr2, color)}
\NormalTok{\{}
  \KeywordTok{points}\NormalTok{(arr1,arr2, }\DataTypeTok{pch=}\DecValTok{7}\NormalTok{, }\DataTypeTok{cex=}\FloatTok{0.5}\NormalTok{, }\DataTypeTok{col =}\NormalTok{ color, }\DataTypeTok{asp=}\DecValTok{1}\NormalTok{,}\DataTypeTok{xlab=}\StringTok{"X"}\NormalTok{, }\DataTypeTok{ylab=}\StringTok{"Y"}\NormalTok{, }\DataTypeTok{main=}\StringTok{"Diagrama "}\NormalTok{)}
\NormalTok{\}}

\CommentTok{# esto, dadas dos listas, encuentra los 3 mejores puntos para spline, tomando los dos extremos y alguno intermedio}
\NormalTok{findMin <-}\StringTok{ }\ControlFlowTok{function}\NormalTok{(lisx, lisy, liminf, limsup)}
\NormalTok{\{}
\NormalTok{  errmin <-}\StringTok{ }\DecValTok{0}
\NormalTok{  pt     <-}\StringTok{ }\DecValTok{0}
\NormalTok{  prim   <-}\StringTok{ }\OtherTok{TRUE}
\NormalTok{  intersec  <-}\StringTok{ }\DecValTok{0}
\NormalTok{  bstintrsc <-}\StringTok{ }\DecValTok{0}
  
  \ControlFlowTok{for}\NormalTok{( i }\ControlFlowTok{in}\NormalTok{ liminf}\OperatorTok{:}\NormalTok{limsup)}
\NormalTok{  \{}
\NormalTok{    valx <-}\StringTok{          }\NormalTok{lisx[liminf]}
\NormalTok{    valx <-}\StringTok{ }\KeywordTok{c}\NormalTok{( valx, lisx[i] )}
\NormalTok{    valx <-}\StringTok{ }\KeywordTok{c}\NormalTok{( valx, lisx[limsup] )}
    
\NormalTok{    valy <-}\StringTok{          }\NormalTok{lisy[liminf]}
\NormalTok{    valy <-}\StringTok{ }\KeywordTok{c}\NormalTok{( valy, lisy[i] )}
\NormalTok{    valy <-}\StringTok{ }\KeywordTok{c}\NormalTok{( valy, lisy[limsup] )}
    
\NormalTok{    sumerr <-}\StringTok{ }\DecValTok{0}    
    
\NormalTok{    ttam <-}\StringTok{ }\NormalTok{limsup }\OperatorTok{-}\StringTok{ }\NormalTok{liminf}
\NormalTok{    ret  <-}\StringTok{ }\KeywordTok{spline}\NormalTok{( valx , valy , }\DataTypeTok{n =}\NormalTok{ ttam , }\DataTypeTok{method =}\NormalTok{ metodo)}
\NormalTok{    ry   <-}\StringTok{ }\NormalTok{ret}\OperatorTok{$}\NormalTok{y}
    
\NormalTok{    intersec <-}\StringTok{ }\DecValTok{0}
    
    \CommentTok{#lines(spline(valx, valy, n = ttam, method = metodo), col = 3)}
    
    \ControlFlowTok{for}\NormalTok{( j }\ControlFlowTok{in} \DecValTok{1}\OperatorTok{:}\NormalTok{ttam )}
\NormalTok{    \{}
      \ControlFlowTok{if}\NormalTok{ (lisy[ liminf }\OperatorTok{+}\StringTok{ }\NormalTok{j ] }\OperatorTok{==}\StringTok{  }\NormalTok{ry[ j ])}
\NormalTok{      \{}
\NormalTok{        intersec <-}\StringTok{ }\NormalTok{intersec }\OperatorTok{+}\StringTok{ }\DecValTok{1}
\NormalTok{      \}}
\NormalTok{      sumerr <-}\StringTok{ }\NormalTok{sumerr }\OperatorTok{+}\StringTok{ }\KeywordTok{abs}\NormalTok{( lisy[ liminf }\OperatorTok{+}\StringTok{ }\NormalTok{j ] }\OperatorTok{-}\StringTok{  }\NormalTok{ry[ j ] ) }
\NormalTok{    \}}
    \ControlFlowTok{if}\NormalTok{ ( }\OperatorTok{!}\KeywordTok{is.nan}\NormalTok{( sumerr ) }\OperatorTok{&&}\StringTok{ }\NormalTok{( prim }\OperatorTok{||}\StringTok{ }\NormalTok{sumerr }\OperatorTok{<}\StringTok{ }\NormalTok{errmin ) )}
\NormalTok{    \{}
\NormalTok{      errmin <-}\StringTok{ }\NormalTok{sumerr}
\NormalTok{      pt     <-}\StringTok{ }\NormalTok{i }
\NormalTok{      prim   <-}\StringTok{ }\OtherTok{FALSE}
\NormalTok{      bstintrsc <-}\StringTok{ }\NormalTok{intersec}
\NormalTok{    \}}
\NormalTok{  \}}
  \KeywordTok{return}\NormalTok{( }\KeywordTok{c}\NormalTok{(pt,errmin,bstintrsc) )}
\NormalTok{\}}

\NormalTok{puntos <-}\StringTok{ }\ControlFlowTok{function}\NormalTok{(x,y)}
\NormalTok{\{}
\NormalTok{  totintersect <-}\StringTok{ }\DecValTok{0}
\NormalTok{  increm <-}\StringTok{ }\FloatTok{1.4}
\NormalTok{  errorinicial   <-}\StringTok{ }\DecValTok{1}
\NormalTok{  erroraceptado  <-}\StringTok{ }\NormalTok{errorinicial}
\NormalTok{  erroraceptadoy <-}\StringTok{ }\NormalTok{errorinicial}
  
\NormalTok{  min    <-}\StringTok{ }\KeywordTok{c}\NormalTok{(}\OtherTok{Inf}\NormalTok{,}\OtherTok{Inf}\NormalTok{,}\OtherTok{Inf}\NormalTok{,}\OtherTok{Inf}\NormalTok{,}\OtherTok{Inf}\NormalTok{)}\CommentTok{# de, medio , hasta, valor, puntos intersectantes }
\NormalTok{  miny    <-}\StringTok{ }\KeywordTok{c}\NormalTok{(}\OtherTok{Inf}\NormalTok{,}\OtherTok{Inf}\NormalTok{,}\OtherTok{Inf}\NormalTok{,}\OtherTok{Inf}\NormalTok{,}\OtherTok{Inf}\NormalTok{)}\CommentTok{# de, medio , hasta, valor, puntos intersectantes }
  
\NormalTok{  i      <-}\StringTok{ }\DecValTok{1}
\NormalTok{  puntos <-}\StringTok{ }\KeywordTok{c}\NormalTok{(}\DecValTok{1}\NormalTok{)}
\NormalTok{  top    <-}\StringTok{ }\DecValTok{1} 
  
\NormalTok{  prim   <-}\StringTok{ }\OtherTok{TRUE}
\NormalTok{  len    <-}\StringTok{ }\DecValTok{0} \CommentTok{# longitud del tramo actual }
  
\NormalTok{  primy   <-}\StringTok{ }\OtherTok{TRUE}
\NormalTok{  leny    <-}\StringTok{ }\DecValTok{0} \CommentTok{# longitud del tramo actual }
  
  \ControlFlowTok{while}\NormalTok{ ( top }\OperatorTok{!=}\StringTok{ }\KeywordTok{length}\NormalTok{(x) )}
\NormalTok{  \{}
    \ControlFlowTok{if}\NormalTok{ ( i }\OperatorTok{-}\StringTok{ }\NormalTok{top }\OperatorTok{>=}\StringTok{ }\DecValTok{1}\NormalTok{)}
\NormalTok{    \{}
\NormalTok{      ret <-}\StringTok{ }\KeywordTok{findMin}\NormalTok{(x,y,top,i) }
\NormalTok{      rety <-}\StringTok{ }\KeywordTok{findMin}\NormalTok{(y,x,top,i) }
      
      \ControlFlowTok{if}\NormalTok{ ( }\KeywordTok{abs}\NormalTok{(min[}\DecValTok{4}\NormalTok{] }\OperatorTok{-}\StringTok{ }\NormalTok{ret[}\DecValTok{2}\NormalTok{]) }\OperatorTok{<=}\StringTok{ }\NormalTok{erroraceptado }\OperatorTok{||}\StringTok{ }\NormalTok{prim) }\CommentTok{# si se encuentran otros valores que minimicen el error, se guardan. o si es el primer valor revisado}
\NormalTok{      \{}
\NormalTok{        min <-}\StringTok{ }\KeywordTok{c}\NormalTok{( top ,ret[}\DecValTok{1}\NormalTok{], i, ret[}\DecValTok{2}\NormalTok{],ret[}\DecValTok{3}\NormalTok{])}
\NormalTok{        prim =}\StringTok{ }\OtherTok{FALSE}
\NormalTok{        erroraceptado <-}\StringTok{ }\NormalTok{erroraceptado }\OperatorTok{+}\StringTok{ }\NormalTok{increm}\OperatorTok{*}\NormalTok{(len}\OperatorTok{/}\KeywordTok{length}\NormalTok{(x)) }\CommentTok{# para darle mas }
\NormalTok{        len <-}\StringTok{ }\DecValTok{1} \CommentTok{# se reinicia la longitud actual }
\NormalTok{      \}}
      \ControlFlowTok{else}
\NormalTok{      \{}
\NormalTok{        len <-}\StringTok{ }\NormalTok{len }\OperatorTok{+}\StringTok{ }\DecValTok{1} \CommentTok{# aumenta la longitud del tramo actual}
\NormalTok{      \}}
      
      \ControlFlowTok{if}\NormalTok{ (}\KeywordTok{abs}\NormalTok{(miny[}\DecValTok{4}\NormalTok{] }\OperatorTok{-}\StringTok{ }\NormalTok{rety[}\DecValTok{2}\NormalTok{]) }\OperatorTok{<=}\StringTok{ }\NormalTok{erroraceptadoy }\OperatorTok{||}\StringTok{ }\NormalTok{primy) }\CommentTok{# si se encuentran otros valores que minimicen el error, se guardan. o si es el primer valor revisado}
\NormalTok{      \{}
\NormalTok{        miny <-}\StringTok{ }\KeywordTok{c}\NormalTok{( top ,rety[}\DecValTok{1}\NormalTok{], i, rety[}\DecValTok{2}\NormalTok{],rety[}\DecValTok{3}\NormalTok{])}
\NormalTok{        primy =}\StringTok{ }\OtherTok{FALSE}
\NormalTok{        erroraceptadoy <-}\StringTok{ }\NormalTok{erroraceptadoy }\OperatorTok{+}\StringTok{ }\NormalTok{increm}\OperatorTok{*}\NormalTok{(len}\OperatorTok{/}\KeywordTok{length}\NormalTok{(x)) }\CommentTok{# para darle mas }
\NormalTok{        leny <-}\StringTok{ }\DecValTok{1} \CommentTok{# se reinicia la longitud actual }
\NormalTok{      \}}
      \ControlFlowTok{else}
\NormalTok{      \{}
\NormalTok{        leny <-}\StringTok{ }\NormalTok{len }\OperatorTok{+}\StringTok{ }\DecValTok{1} \CommentTok{# aumenta la longitud del tramo actual}
\NormalTok{      \}}
      
      \ControlFlowTok{if}\NormalTok{ ( i }\OperatorTok{>=}\StringTok{ }\KeywordTok{length}\NormalTok{(x) )}
\NormalTok{      \{}
        \CommentTok{#cat ("\textbackslash{}n\textbackslash{}n\textbackslash{}n",min,"      ",miny ,"\textbackslash{}n")}
        
        \ControlFlowTok{if}\NormalTok{ (miny[}\DecValTok{4}\NormalTok{] }\OperatorTok{+}\StringTok{ }\NormalTok{(leny}\OperatorTok{/}\KeywordTok{length}\NormalTok{(x)) }\OperatorTok{<}\StringTok{ }\NormalTok{min[}\DecValTok{4}\NormalTok{])}
\NormalTok{        \{}
          \KeywordTok{cat}\NormalTok{(}\StringTok{"y "}\NormalTok{)}
\NormalTok{          min <-}\StringTok{ }\NormalTok{miny}
\NormalTok{          color <-}\StringTok{ }\DecValTok{3}
\NormalTok{        \}}
        \ControlFlowTok{else}
\NormalTok{        \{}
          \KeywordTok{cat}\NormalTok{(}\StringTok{"x "}\NormalTok{)}
\NormalTok{          color <-}\StringTok{ }\DecValTok{2}
\NormalTok{        \}}
        
\NormalTok{        nx <-}\StringTok{ }\KeywordTok{c}\NormalTok{( x[ min[}\DecValTok{1}\NormalTok{] ], x[ min[}\DecValTok{2}\NormalTok{] ] , x[ min[}\DecValTok{3}\NormalTok{] ] )}
\NormalTok{        ny <-}\StringTok{ }\KeywordTok{c}\NormalTok{( y[ min[}\DecValTok{1}\NormalTok{] ], y[ min[}\DecValTok{2}\NormalTok{] ] , y[ min[}\DecValTok{3}\NormalTok{] ] )}
        \KeywordTok{lines}\NormalTok{(}\KeywordTok{spline}\NormalTok{(nx, ny, }\DataTypeTok{n =} \DecValTok{200}\NormalTok{, }\DataTypeTok{method =}\NormalTok{ metodo ), }\DataTypeTok{col =}\NormalTok{ color)}
        
\NormalTok{        erroraceptado  <-}\StringTok{ }\NormalTok{errorinicial }\CommentTok{# reiniciar el error aceptado}
\NormalTok{        erroraceptadoy <-}\StringTok{ }\NormalTok{errorinicial }
        
        \KeywordTok{cat}\NormalTok{(}\StringTok{"desde : "}\NormalTok{,min[}\DecValTok{1}\NormalTok{],}\StringTok{"}\CharTok{\textbackslash{}t}\StringTok{punto medio : "}\NormalTok{,min[}\DecValTok{2}\NormalTok{], }\StringTok{"}\CharTok{\textbackslash{}t}\StringTok{hasta : "}\NormalTok{,min[}\DecValTok{3}\NormalTok{], }\StringTok{"}\CharTok{\textbackslash{}t}\StringTok{puntos intersectantes : "}\NormalTok{,min[}\DecValTok{5}\NormalTok{],}\StringTok{"}\CharTok{\textbackslash{}t}\StringTok{error segmento : "}\NormalTok{,min[}\DecValTok{4}\NormalTok{],}\StringTok{"}\CharTok{\textbackslash{}n}\StringTok{"}\NormalTok{)}
\NormalTok{        totintersect <-}\StringTok{ }\NormalTok{totintersect }\OperatorTok{+}\StringTok{ }\NormalTok{min[}\DecValTok{5}\NormalTok{]}
        \ControlFlowTok{if}\NormalTok{ ( min[}\DecValTok{2}\NormalTok{] }\OperatorTok{==}\StringTok{ }\NormalTok{puntos[ }\KeywordTok{length}\NormalTok{( puntos ) ] ) }
\NormalTok{        \{}
\NormalTok{          puntos <-}\StringTok{ }\KeywordTok{c}\NormalTok{( puntos,min[ }\DecValTok{3}\NormalTok{ ] ) }\CommentTok{# min[1] en teoria es el ultimo de los anteriores, entonces ya esta }
\NormalTok{        \}}
        \ControlFlowTok{else}
\NormalTok{        \{}
\NormalTok{          puntos <-}\StringTok{ }\KeywordTok{c}\NormalTok{( puntos,min[ }\DecValTok{2}\OperatorTok{:}\DecValTok{3}\NormalTok{ ] )}
\NormalTok{        \}}
        
\NormalTok{        top <-}\StringTok{ }\NormalTok{min[}\DecValTok{3}\NormalTok{]}
\NormalTok{        i   <-}\StringTok{ }\NormalTok{top}
        
\NormalTok{        min  <-}\StringTok{ }\KeywordTok{c}\NormalTok{(}\OtherTok{Inf}\NormalTok{,}\OtherTok{Inf}\NormalTok{,}\OtherTok{Inf}\NormalTok{,}\OtherTok{Inf}\NormalTok{,}\OtherTok{Inf}\NormalTok{)}\CommentTok{# de, medio , hasta, valor, interseccion}
\NormalTok{        prim <-}\StringTok{ }\OtherTok{TRUE}
        

        
\NormalTok{      \}}
      \ControlFlowTok{else} 
\NormalTok{      \{}
\NormalTok{        i <-}\StringTok{ }\NormalTok{i }\OperatorTok{+}\StringTok{ }\DecValTok{1}  
\NormalTok{      \}}
\NormalTok{    \}}
    \ControlFlowTok{else} 
\NormalTok{    \{}
\NormalTok{      i <-}\StringTok{ }\NormalTok{i }\OperatorTok{+}\StringTok{ }\DecValTok{1}  
\NormalTok{    \}}
    
    \ControlFlowTok{if}\NormalTok{ ( i }\OperatorTok{>=}\StringTok{ }\KeywordTok{length}\NormalTok{(x) ) }\CommentTok{# por si acaso :v }
\NormalTok{    \{}
\NormalTok{      i <-}\StringTok{ }\KeywordTok{length}\NormalTok{(x)}
\NormalTok{    \}}
\NormalTok{  \}}
  \KeywordTok{cat}\NormalTok{(}\StringTok{"}\CharTok{\textbackslash{}n\textbackslash{}n}\StringTok{cantidad : "}\NormalTok{,}\KeywordTok{length}\NormalTok{(puntos),}\StringTok{"}\CharTok{\textbackslash{}t}\StringTok{indice de Jaccard : "}\NormalTok{,totintersect}\OperatorTok{/}\KeywordTok{length}\NormalTok{(x),}\StringTok{"}\CharTok{\textbackslash{}n}\StringTok{puntos seleccionados : "}\NormalTok{,puntos,}\StringTok{"}\CharTok{\textbackslash{}n}\StringTok{"}\NormalTok{)}
\NormalTok{\}}


\CommentTok{# reorganizacion de datos }
\NormalTok{x =}\StringTok{ }\KeywordTok{c}\NormalTok{(}\FloatTok{14.65}\NormalTok{, }\FloatTok{14.71}\NormalTok{, }\FloatTok{14.6}\NormalTok{, }\FloatTok{14.8}\NormalTok{, }\FloatTok{15.2}\NormalTok{, }\FloatTok{15.6}\NormalTok{, }\FloatTok{15.7}\NormalTok{, }\FloatTok{17.0}\NormalTok{, }\FloatTok{17.6}\NormalTok{, }\FloatTok{17.52}\NormalTok{, }\FloatTok{17.3}\NormalTok{, }
      \FloatTok{16.8}\NormalTok{, }\FloatTok{15.4}\NormalTok{, }\FloatTok{14.83}\NormalTok{, }\FloatTok{14.4}\NormalTok{, }\FloatTok{14.5}\NormalTok{, }
      \FloatTok{15.0}\NormalTok{, }\FloatTok{15.1}\NormalTok{, }\FloatTok{15.0}\NormalTok{, }\FloatTok{14.9}\NormalTok{, }\FloatTok{14.6}\NormalTok{, }\FloatTok{14.3}\NormalTok{, }\FloatTok{14.0}\NormalTok{, }\FloatTok{13.9}\NormalTok{, }\FloatTok{13.8}\NormalTok{, }\FloatTok{13.5}\NormalTok{, }\FloatTok{13.1}\NormalTok{, }\FloatTok{13.0}\NormalTok{, }
      \FloatTok{13.3}\NormalTok{, }\FloatTok{13.2}\NormalTok{, }\FloatTok{13.1}\NormalTok{, }\FloatTok{12.9}\NormalTok{, }\FloatTok{12.4}\NormalTok{, }\FloatTok{11.9}\NormalTok{, }\FloatTok{11.7}\NormalTok{, }\FloatTok{11.6}\NormalTok{, }\FloatTok{11.3}\NormalTok{, }\FloatTok{10.9}\NormalTok{, }
      \FloatTok{10.7}\NormalTok{, }\FloatTok{10.6}\NormalTok{, }\FloatTok{10.6}\NormalTok{, }\FloatTok{10.1}\NormalTok{, }\FloatTok{9.7}\NormalTok{, }\FloatTok{9.4}\NormalTok{, }\FloatTok{9.3}\NormalTok{, }\FloatTok{9.6}\NormalTok{, }\FloatTok{9.9}\NormalTok{, }\FloatTok{10.1}\NormalTok{, }\FloatTok{10.2}\NormalTok{, }\FloatTok{10.3}\NormalTok{,  }\FloatTok{10.0}\NormalTok{, }\FloatTok{9.5}\NormalTok{, }
      \FloatTok{8.6}\NormalTok{, }\FloatTok{7.5}\NormalTok{, }\FloatTok{7.0}\NormalTok{, }\FloatTok{6.7}\NormalTok{, }\FloatTok{6.6}\NormalTok{, }\FloatTok{7.7}\NormalTok{, }
      \FloatTok{8.0}\NormalTok{, }\FloatTok{8.1}\NormalTok{, }\FloatTok{8.4}\NormalTok{, }\FloatTok{9.2}\NormalTok{, }\FloatTok{9.3}\NormalTok{, }\DecValTok{10}\NormalTok{, }\FloatTok{10.2}\NormalTok{, }\FloatTok{10.3}\NormalTok{)}

\NormalTok{y =}\StringTok{ }\KeywordTok{c}\NormalTok{(}\FloatTok{14.7}\NormalTok{, }\FloatTok{14.33}\NormalTok{, }\FloatTok{13.4}\NormalTok{, }\FloatTok{12.33}\NormalTok{, }\FloatTok{11.0}\NormalTok{, }\FloatTok{10.5}\NormalTok{, }\FloatTok{10.22}\NormalTok{, }\FloatTok{8.2}\NormalTok{, }\FloatTok{7.1}\NormalTok{, }\FloatTok{6.7}\NormalTok{, }\FloatTok{6.6}\NormalTok{, }
      \FloatTok{6.8}\NormalTok{, }\FloatTok{8.3}\NormalTok{, }\FloatTok{8.8}\NormalTok{, }\FloatTok{9.3}\NormalTok{, }\FloatTok{8.8}\NormalTok{, }
      \FloatTok{6.3}\NormalTok{, }\FloatTok{5.5}\NormalTok{, }\FloatTok{5.0}\NormalTok{, }\FloatTok{4.7}\NormalTok{, }\FloatTok{4.6}\NormalTok{, }\FloatTok{4.5}\NormalTok{, }\FloatTok{4.9}\NormalTok{, }\FloatTok{5.4}\NormalTok{, }\FloatTok{5.8}\NormalTok{, }\FloatTok{6.9}\NormalTok{, }\FloatTok{8.2}\NormalTok{, }\FloatTok{7.6}\NormalTok{, }
      \FloatTok{5.8}\NormalTok{, }\FloatTok{4.5}\NormalTok{, }\FloatTok{4.3}\NormalTok{, }\FloatTok{3.9}\NormalTok{, }\FloatTok{4.2}\NormalTok{, }\FloatTok{5.7}\NormalTok{, }\FloatTok{7.0}\NormalTok{, }\FloatTok{7.9}\NormalTok{, }\FloatTok{8.2}\NormalTok{, }\FloatTok{7.3}\NormalTok{, }
      \FloatTok{6.7}\NormalTok{, }\FloatTok{5.5}\NormalTok{, }\FloatTok{5.1}\NormalTok{, }\FloatTok{4.6}\NormalTok{, }\FloatTok{4.7}\NormalTok{, }\FloatTok{5.0}\NormalTok{, }\FloatTok{5.5}\NormalTok{, }\FloatTok{7.2}\NormalTok{, }\FloatTok{7.8}\NormalTok{, }\FloatTok{8.6}\NormalTok{, }\FloatTok{9.4}\NormalTok{, }\FloatTok{10.0}\NormalTok{,  }\FloatTok{10.7}\NormalTok{, }\FloatTok{11.0}\NormalTok{, }
      \FloatTok{9.9}\NormalTok{, }\FloatTok{9.0}\NormalTok{, }\FloatTok{9.1}\NormalTok{, }\FloatTok{9.3}\NormalTok{, }\FloatTok{9.7}\NormalTok{, }\FloatTok{11.7}\NormalTok{, }
      \FloatTok{12.3}\NormalTok{, }\FloatTok{12.5}\NormalTok{, }\FloatTok{13.0}\NormalTok{, }\FloatTok{13.91}\NormalTok{, }\FloatTok{14.9}\NormalTok{, }\DecValTok{16}\NormalTok{, }\FloatTok{16.4}\NormalTok{, }\FloatTok{16.8}\NormalTok{)}


\KeywordTok{plot}\NormalTok{(x,y,}\DataTypeTok{type=}\StringTok{"l"}\NormalTok{,}\DataTypeTok{asp=}\DecValTok{1}\NormalTok{,}\DataTypeTok{main=}\StringTok{"Mano"}\NormalTok{)}

\KeywordTok{puntos}\NormalTok{(x,y)}
\end{Highlighting}
\end{Shaded}

\includegraphics{Documento_notebook_files/figure-latex/unnamed-chunk-7-1.pdf}

\begin{verbatim}
## y desde :  1     punto medio :  3    hasta :  4  puntos intersectantes :  0  error segmento :  0.2485361 
## y desde :  4     punto medio :  4    hasta :  5  puntos intersectantes :  1  error segmento :  0 
## y desde :  5     punto medio :  5    hasta :  7  puntos intersectantes :  0  error segmento :  0.6 
## y desde :  7     punto medio :  7    hasta :  8  puntos intersectantes :  1  error segmento :  0 
## x desde :  8     punto medio :  10   hasta :  11     puntos intersectantes :  0  error segmento :  1.202424 
## x desde :  11    punto medio :  11   hasta :  12     puntos intersectantes :  1  error segmento :  0 
## x desde :  12    punto medio :  12   hasta :  13     puntos intersectantes :  1  error segmento :  0 
## x desde :  13    punto medio :  14   hasta :  16     puntos intersectantes :  1  error segmento :  1.052632 
## x desde :  16    punto medio :  19   hasta :  20     puntos intersectantes :  0  error segmento :  3.566667 
## x desde :  20    punto medio :  21   hasta :  25     puntos intersectantes :  0  error segmento :  3.904167 
## x desde :  25    punto medio :  25   hasta :  26     puntos intersectantes :  1  error segmento :  0 
## x desde :  26    punto medio :  27   hasta :  29     puntos intersectantes :  1  error segmento :  2.9 
## x desde :  29    punto medio :  33   hasta :  34     puntos intersectantes :  0  error segmento :  1.969444 
## x desde :  34    punto medio :  37   hasta :  39     puntos intersectantes :  1  error segmento :  1.5 
## x desde :  39    punto medio :  39   hasta :  40     puntos intersectantes :  1  error segmento :  0 
## x desde :  40    punto medio :  43   hasta :  45     puntos intersectantes :  1  error segmento :  0.9388889 
## x desde :  45    punto medio :  46   hasta :  50     puntos intersectantes :  1  error segmento :  3.208333 
## x desde :  50    punto medio :  52   hasta :  53     puntos intersectantes :  0  error segmento :  0.9006944 
## x desde :  53    punto medio :  54   hasta :  57     puntos intersectantes :  0  error segmento :  1.118631 
## x desde :  57    punto medio :  63   hasta :  65     puntos intersectantes :  1  error segmento :  5.243333 
## x desde :  65    punto medio :  65   hasta :  66     puntos intersectantes :  0  error segmento :  0.4 
## 
## 
## cantidad :  35   indice de Jaccard :  0.1818182 
## puntos seleccionados :  1 3 4 5 7 8 10 11 12 13 14 16 19 20 21 25 26 27 29 33 34 37 39 40 43 45 46 50 52 53 54 57 63 65 66
\end{verbatim}

Como se puede observar en la gráfica se tiene un tramo de color verde,
este tramo presenta la variación de x y y mencionada anteriormente; por
otro lado este algoritmo termina con 35 puntos utilizados, a diferencia
del anterior utiliza 5 más por la variación

\subsection{Solución preguntas}\label{solucion-preguntas}

-¿El origen se puede modificar? Si tenemos nueva información (nodos)
¿como podemos implementar esa información en el algoritmo de
interpolación?\\
Debido a que el método utilizado se basa en ir verificando errores entre
los diferentes puntos que se encuentran en el vector inicial y el spline
calculado, es decir no depende del origen, al cambiar el valor inicial o
el origen el sistema solamente se vera afectado al inicio de la
interpolación de la mano de resto no se verá afectado debido a que
seguirá verificando errores con dicha función hasta encontrar el menor
error y seguirá utilizando los segmentos de menor error con los puntos
que estos establezcan.

Se puede observar con el siguiente ejemplo: Si se quita el primer dato
de la mano se genera una deformación en el dedo meñique (primer dedo que
gráfica)

\begin{Shaded}
\begin{Highlighting}[]
\KeywordTok{rm}\NormalTok{(}\DataTypeTok{list=}\KeywordTok{ls}\NormalTok{())}

\NormalTok{metodo <-}\StringTok{ "fmm"}  \CommentTok{# metodos : ("fmm","natural","periodic") }

\NormalTok{graf <-}\StringTok{ }\ControlFlowTok{function}\NormalTok{(arr1,arr2, color)}
\NormalTok{\{}
  \KeywordTok{points}\NormalTok{(arr1,arr2, }\DataTypeTok{pch=}\DecValTok{7}\NormalTok{, }\DataTypeTok{cex=}\FloatTok{0.5}\NormalTok{, }\DataTypeTok{col =}\NormalTok{ color, }\DataTypeTok{asp=}\DecValTok{1}\NormalTok{,}\DataTypeTok{xlab=}\StringTok{"X"}\NormalTok{, }\DataTypeTok{ylab=}\StringTok{"Y"}\NormalTok{, }\DataTypeTok{main=}\StringTok{"Diagrama "}\NormalTok{)}
\NormalTok{\}}

\CommentTok{# esto, dada una lista, encuentra los 3 mejores puntos para lagrange, tomando los dos extremos y alguno intermedio}
\NormalTok{findMin <-}\StringTok{ }\ControlFlowTok{function}\NormalTok{(lisx, lisy, liminf, limsup)}
\NormalTok{\{}
\NormalTok{  errmin <-}\StringTok{ }\DecValTok{0}
\NormalTok{  pt     <-}\StringTok{ }\DecValTok{0}
\NormalTok{  prim   <-}\StringTok{ }\OtherTok{TRUE}
\NormalTok{  intersec  <-}\StringTok{ }\DecValTok{0}
\NormalTok{  bstintrsc <-}\StringTok{ }\DecValTok{0}
  
  \ControlFlowTok{for}\NormalTok{( i }\ControlFlowTok{in}\NormalTok{ liminf}\OperatorTok{:}\NormalTok{limsup)}
\NormalTok{  \{}
\NormalTok{    valx <-}\StringTok{          }\NormalTok{lisx[liminf]}
\NormalTok{    valx <-}\StringTok{ }\KeywordTok{c}\NormalTok{( valx, lisx[i] )}
\NormalTok{    valx <-}\StringTok{ }\KeywordTok{c}\NormalTok{( valx, lisx[limsup] )}
    
\NormalTok{    valy <-}\StringTok{          }\NormalTok{lisy[liminf]}
\NormalTok{    valy <-}\StringTok{ }\KeywordTok{c}\NormalTok{( valy, lisy[i] )}
\NormalTok{    valy <-}\StringTok{ }\KeywordTok{c}\NormalTok{( valy, lisy[limsup] )}
    
\NormalTok{    sumerr <-}\StringTok{ }\DecValTok{0}    
    
\NormalTok{    ttam <-}\StringTok{ }\NormalTok{limsup }\OperatorTok{-}\StringTok{ }\NormalTok{liminf}
\NormalTok{    ret  <-}\StringTok{ }\KeywordTok{spline}\NormalTok{( valx , valy , }\DataTypeTok{n =}\NormalTok{ ttam , }\DataTypeTok{method =}\NormalTok{ metodo)}
\NormalTok{    ry   <-}\StringTok{ }\NormalTok{ret}\OperatorTok{$}\NormalTok{y}
    
\NormalTok{    intersec <-}\StringTok{ }\DecValTok{0}
    
    \CommentTok{#lines(spline(valx, valy, n = ttam, method = metodo), col = 3)}
    
    \ControlFlowTok{for}\NormalTok{( j }\ControlFlowTok{in} \DecValTok{1}\OperatorTok{:}\NormalTok{ttam )}
\NormalTok{    \{}
      \ControlFlowTok{if}\NormalTok{ (lisy[ liminf }\OperatorTok{+}\StringTok{ }\NormalTok{j ] }\OperatorTok{==}\StringTok{  }\NormalTok{ry[ j ])}
\NormalTok{      \{}
\NormalTok{        intersec <-}\StringTok{ }\NormalTok{intersec }\OperatorTok{+}\StringTok{ }\DecValTok{1}
        \CommentTok{#cat (lisy[ liminf + j ]  ,"-----",  ry[ j ],"\textbackslash{}n")}
\NormalTok{      \}}
\NormalTok{      sumerr <-}\StringTok{ }\NormalTok{sumerr }\OperatorTok{+}\StringTok{ }\KeywordTok{abs}\NormalTok{( lisy[ liminf }\OperatorTok{+}\StringTok{ }\NormalTok{j ] }\OperatorTok{-}\StringTok{  }\NormalTok{ry[ j ] ) }
\NormalTok{    \}}
    \ControlFlowTok{if}\NormalTok{ ( }\OperatorTok{!}\KeywordTok{is.nan}\NormalTok{( sumerr ) }\OperatorTok{&&}\StringTok{ }\NormalTok{( prim }\OperatorTok{||}\StringTok{ }\NormalTok{sumerr }\OperatorTok{<}\StringTok{ }\NormalTok{errmin ) )}
\NormalTok{    \{}
\NormalTok{      errmin <-}\StringTok{ }\NormalTok{sumerr}
\NormalTok{      pt     <-}\StringTok{ }\NormalTok{i }
\NormalTok{      prim   <-}\StringTok{ }\OtherTok{FALSE}
\NormalTok{      bstintrsc <-}\StringTok{ }\NormalTok{intersec}
\NormalTok{    \}}
\NormalTok{  \}}
  \KeywordTok{return}\NormalTok{( }\KeywordTok{c}\NormalTok{(pt,errmin,bstintrsc) )}
\NormalTok{\}}

\NormalTok{puntos <-}\StringTok{ }\ControlFlowTok{function}\NormalTok{(x,y)}
\NormalTok{\{}
\NormalTok{  totintersect <-}\StringTok{ }\DecValTok{0}
\NormalTok{  increm <-}\StringTok{ }\FloatTok{1.4}
\NormalTok{  errorinicial  <-}\StringTok{ }\DecValTok{1}
\NormalTok{  erroraceptado <-}\StringTok{ }\NormalTok{errorinicial}
\NormalTok{  puntos <-}\StringTok{ }\KeywordTok{c}\NormalTok{(}\DecValTok{1}\NormalTok{)}
\NormalTok{  top    <-}\StringTok{ }\DecValTok{1} 
\NormalTok{  min    <-}\StringTok{ }\KeywordTok{c}\NormalTok{(}\OtherTok{Inf}\NormalTok{,}\OtherTok{Inf}\NormalTok{,}\OtherTok{Inf}\NormalTok{,}\OtherTok{Inf}\NormalTok{,}\OtherTok{Inf}\NormalTok{)}\CommentTok{# de, medio , hasta, valor, puntos intersectantes }
\NormalTok{  i      <-}\StringTok{ }\DecValTok{3}
\NormalTok{  prim   <-}\StringTok{ }\OtherTok{TRUE}
\NormalTok{  len    <-}\StringTok{ }\DecValTok{0} \CommentTok{# longitud del tramo actual }
  
  \ControlFlowTok{while}\NormalTok{ ( top }\OperatorTok{!=}\StringTok{ }\KeywordTok{length}\NormalTok{(x) )}
\NormalTok{  \{}
    \ControlFlowTok{if}\NormalTok{ ( i }\OperatorTok{-}\StringTok{ }\NormalTok{top }\OperatorTok{>=}\StringTok{ }\DecValTok{1}\NormalTok{)}
\NormalTok{    \{}
\NormalTok{      ret <-}\StringTok{ }\KeywordTok{findMin}\NormalTok{(x,y,top,i) }
      
      \ControlFlowTok{if}\NormalTok{ ( }\KeywordTok{abs}\NormalTok{(min[}\DecValTok{4}\NormalTok{] }\OperatorTok{-}\StringTok{ }\NormalTok{ret[}\DecValTok{2}\NormalTok{]) }\OperatorTok{<=}\StringTok{ }\NormalTok{erroraceptado }\OperatorTok{||}\StringTok{ }\NormalTok{prim) }\CommentTok{# si se encuentran otros valores que minimicen el error, se guardan. o si es el primer valor revisado}
\NormalTok{      \{}
\NormalTok{        min <-}\StringTok{ }\KeywordTok{c}\NormalTok{( top ,ret[}\DecValTok{1}\NormalTok{], i, ret[}\DecValTok{2}\NormalTok{],ret[}\DecValTok{3}\NormalTok{])}
\NormalTok{        prim =}\StringTok{ }\OtherTok{FALSE}
\NormalTok{        erroraceptado <-}\StringTok{ }\NormalTok{erroraceptado }\OperatorTok{+}\StringTok{ }\NormalTok{increm}\OperatorTok{*}\NormalTok{(len}\OperatorTok{/}\KeywordTok{length}\NormalTok{(x)) }\CommentTok{# para darle mas }
\NormalTok{        len <-}\StringTok{ }\DecValTok{1} \CommentTok{# se reinicia la longitud actual }
\NormalTok{      \}}
      \ControlFlowTok{else}
\NormalTok{      \{}
\NormalTok{        len <-}\StringTok{ }\NormalTok{len }\OperatorTok{+}\StringTok{ }\DecValTok{1} \CommentTok{# aumenta la longitud del tramo actual}
\NormalTok{      \}}
      
      \ControlFlowTok{if}\NormalTok{ ( i }\OperatorTok{>=}\StringTok{ }\KeywordTok{length}\NormalTok{(x) )}
\NormalTok{      \{}
\NormalTok{        nx <-}\StringTok{ }\KeywordTok{c}\NormalTok{( x[ min[}\DecValTok{1}\NormalTok{] ], x[ min[}\DecValTok{2}\NormalTok{] ] , x[ min[}\DecValTok{3}\NormalTok{] ] )}
\NormalTok{        ny <-}\StringTok{ }\KeywordTok{c}\NormalTok{( y[ min[}\DecValTok{1}\NormalTok{] ], y[ min[}\DecValTok{2}\NormalTok{] ] , y[ min[}\DecValTok{3}\NormalTok{] ] )}
        \KeywordTok{lines}\NormalTok{(}\KeywordTok{spline}\NormalTok{(nx, ny, }\DataTypeTok{n =} \DecValTok{200}\NormalTok{, }\DataTypeTok{method =}\NormalTok{ metodo ), }\DataTypeTok{col =} \DecValTok{2}\NormalTok{)}
        
\NormalTok{        erroraceptado <-}\StringTok{ }\NormalTok{errorinicial }\CommentTok{# reiniciar el error aceptado}
        
        \KeywordTok{cat}\NormalTok{(}\StringTok{"desde : "}\NormalTok{,min[}\DecValTok{1}\NormalTok{],}\StringTok{"}\CharTok{\textbackslash{}t}\StringTok{punto medio : "}\NormalTok{,min[}\DecValTok{2}\NormalTok{], }\StringTok{"}\CharTok{\textbackslash{}t}\StringTok{hasta : "}\NormalTok{,min[}\DecValTok{3}\NormalTok{], }\StringTok{"}\CharTok{\textbackslash{}t}\StringTok{puntos intersectantes : "}\NormalTok{,min[}\DecValTok{5}\NormalTok{],}\StringTok{"}\CharTok{\textbackslash{}t}\StringTok{error segmento : "}\NormalTok{,min[}\DecValTok{4}\NormalTok{],}\StringTok{"}\CharTok{\textbackslash{}n}\StringTok{"}\NormalTok{)}
\NormalTok{        totintersect <-}\StringTok{ }\NormalTok{totintersect }\OperatorTok{+}\StringTok{ }\NormalTok{min[}\DecValTok{5}\NormalTok{]}
        \ControlFlowTok{if}\NormalTok{ ( min[}\DecValTok{2}\NormalTok{] }\OperatorTok{==}\StringTok{ }\NormalTok{puntos[ }\KeywordTok{length}\NormalTok{( puntos ) ] ) }
\NormalTok{        \{}
\NormalTok{          puntos <-}\StringTok{ }\KeywordTok{c}\NormalTok{( puntos,min[ }\DecValTok{3}\NormalTok{ ] ) }\CommentTok{# min[1] en teoria es el ultimo de los anteriores, entonces ya esta }
\NormalTok{        \}}
        \ControlFlowTok{else}
\NormalTok{        \{}
\NormalTok{          puntos <-}\StringTok{ }\KeywordTok{c}\NormalTok{( puntos,min[ }\DecValTok{2}\OperatorTok{:}\DecValTok{3}\NormalTok{ ] )}
\NormalTok{        \}}
        
\NormalTok{        top <-}\StringTok{ }\NormalTok{min[}\DecValTok{3}\NormalTok{]}
\NormalTok{        i   <-}\StringTok{ }\NormalTok{top}
        
\NormalTok{        min  <-}\StringTok{ }\KeywordTok{c}\NormalTok{(}\OtherTok{Inf}\NormalTok{,}\OtherTok{Inf}\NormalTok{,}\OtherTok{Inf}\NormalTok{,}\OtherTok{Inf}\NormalTok{,}\OtherTok{Inf}\NormalTok{)}\CommentTok{# de, medio , hasta, valor, interseccion}
\NormalTok{        prim <-}\StringTok{ }\OtherTok{TRUE}
\NormalTok{      \}}
      \ControlFlowTok{else} 
\NormalTok{      \{}
\NormalTok{        i <-}\StringTok{ }\NormalTok{i }\OperatorTok{+}\StringTok{ }\DecValTok{1}  
\NormalTok{      \}}
\NormalTok{    \}}
    \ControlFlowTok{else} 
\NormalTok{    \{}
\NormalTok{      i <-}\StringTok{ }\NormalTok{i }\OperatorTok{+}\StringTok{ }\DecValTok{1}  
\NormalTok{    \}}
    
    \ControlFlowTok{if}\NormalTok{ ( i }\OperatorTok{>=}\StringTok{ }\KeywordTok{length}\NormalTok{(x) ) }\CommentTok{# por si acaso :v }
\NormalTok{    \{}
\NormalTok{      i <-}\StringTok{ }\KeywordTok{length}\NormalTok{(x)}
\NormalTok{    \}}
\NormalTok{  \}}
  \KeywordTok{cat}\NormalTok{(}\StringTok{"}\CharTok{\textbackslash{}n\textbackslash{}n}\StringTok{cantidad : "}\NormalTok{,}\KeywordTok{length}\NormalTok{(puntos),}\StringTok{"}\CharTok{\textbackslash{}t}\StringTok{indice de Jaccard : "}\NormalTok{,totintersect}\OperatorTok{/}\KeywordTok{length}\NormalTok{(x),}\StringTok{"}\CharTok{\textbackslash{}n}\StringTok{puntos seleccionados : "}\NormalTok{,puntos,}\StringTok{"}\CharTok{\textbackslash{}n}\StringTok{"}\NormalTok{)}
\NormalTok{\}}

\CommentTok{# reorganizacion de datos }
\NormalTok{x =}\StringTok{ }\KeywordTok{c}\NormalTok{(}\FloatTok{14.71}\NormalTok{, }\FloatTok{14.6}\NormalTok{, }\FloatTok{14.8}\NormalTok{, }\FloatTok{15.2}\NormalTok{, }\FloatTok{15.6}\NormalTok{, }\FloatTok{15.7}\NormalTok{, }\FloatTok{17.0}\NormalTok{, }\FloatTok{17.6}\NormalTok{, }\FloatTok{17.52}\NormalTok{, }\FloatTok{17.3}\NormalTok{, }
  \FloatTok{16.8}\NormalTok{, }\FloatTok{15.4}\NormalTok{, }\FloatTok{14.83}\NormalTok{, }\FloatTok{14.4}\NormalTok{, }\FloatTok{14.5}\NormalTok{, }
  \FloatTok{15.0}\NormalTok{, }\FloatTok{15.1}\NormalTok{, }\FloatTok{15.0}\NormalTok{, }\FloatTok{14.9}\NormalTok{, }\FloatTok{14.6}\NormalTok{, }\FloatTok{14.3}\NormalTok{, }\FloatTok{14.0}\NormalTok{, }\FloatTok{13.9}\NormalTok{, }\FloatTok{13.8}\NormalTok{, }\FloatTok{13.5}\NormalTok{, }\FloatTok{13.1}\NormalTok{, }\FloatTok{13.0}\NormalTok{, }
  \FloatTok{13.3}\NormalTok{, }\FloatTok{13.2}\NormalTok{, }\FloatTok{13.1}\NormalTok{, }\FloatTok{12.9}\NormalTok{, }\FloatTok{12.4}\NormalTok{, }\FloatTok{11.9}\NormalTok{, }\FloatTok{11.7}\NormalTok{, }\FloatTok{11.6}\NormalTok{, }\FloatTok{11.3}\NormalTok{, }\FloatTok{10.9}\NormalTok{, }
  \FloatTok{10.7}\NormalTok{, }\FloatTok{10.6}\NormalTok{, }\FloatTok{10.6}\NormalTok{, }\FloatTok{10.1}\NormalTok{, }\FloatTok{9.7}\NormalTok{, }\FloatTok{9.4}\NormalTok{, }\FloatTok{9.3}\NormalTok{, }\FloatTok{9.6}\NormalTok{, }\FloatTok{9.9}\NormalTok{, }\FloatTok{10.1}\NormalTok{, }\FloatTok{10.2}\NormalTok{, }\FloatTok{10.3}\NormalTok{,  }\FloatTok{10.0}\NormalTok{, }\FloatTok{9.5}\NormalTok{, }
  \FloatTok{8.6}\NormalTok{, }\FloatTok{7.5}\NormalTok{, }\FloatTok{7.0}\NormalTok{, }\FloatTok{6.7}\NormalTok{, }\FloatTok{6.6}\NormalTok{, }\FloatTok{7.7}\NormalTok{, }
  \FloatTok{8.0}\NormalTok{, }\FloatTok{8.1}\NormalTok{, }\FloatTok{8.4}\NormalTok{, }\FloatTok{9.2}\NormalTok{, }\FloatTok{9.3}\NormalTok{, }\DecValTok{10}\NormalTok{, }\FloatTok{10.2}\NormalTok{, }\FloatTok{10.3}\NormalTok{)}

\NormalTok{y =}\StringTok{ }\KeywordTok{c}\NormalTok{( }\FloatTok{14.33}\NormalTok{, }\FloatTok{13.4}\NormalTok{, }\FloatTok{12.33}\NormalTok{, }\FloatTok{11.0}\NormalTok{, }\FloatTok{10.5}\NormalTok{, }\FloatTok{10.22}\NormalTok{, }\FloatTok{8.2}\NormalTok{, }\FloatTok{7.1}\NormalTok{, }\FloatTok{6.7}\NormalTok{, }\FloatTok{6.6}\NormalTok{, }
  \FloatTok{6.8}\NormalTok{, }\FloatTok{8.3}\NormalTok{, }\FloatTok{8.8}\NormalTok{, }\FloatTok{9.3}\NormalTok{, }\FloatTok{8.8}\NormalTok{, }
  \FloatTok{6.3}\NormalTok{, }\FloatTok{5.5}\NormalTok{, }\FloatTok{5.0}\NormalTok{, }\FloatTok{4.7}\NormalTok{, }\FloatTok{4.6}\NormalTok{, }\FloatTok{4.5}\NormalTok{, }\FloatTok{4.9}\NormalTok{, }\FloatTok{5.4}\NormalTok{, }\FloatTok{5.8}\NormalTok{, }\FloatTok{6.9}\NormalTok{, }\FloatTok{8.2}\NormalTok{, }\FloatTok{7.6}\NormalTok{, }
  \FloatTok{5.8}\NormalTok{, }\FloatTok{4.5}\NormalTok{, }\FloatTok{4.3}\NormalTok{, }\FloatTok{3.9}\NormalTok{, }\FloatTok{4.2}\NormalTok{, }\FloatTok{5.7}\NormalTok{, }\FloatTok{7.0}\NormalTok{, }\FloatTok{7.9}\NormalTok{, }\FloatTok{8.2}\NormalTok{, }\FloatTok{7.3}\NormalTok{, }
  \FloatTok{6.7}\NormalTok{, }\FloatTok{5.5}\NormalTok{, }\FloatTok{5.1}\NormalTok{, }\FloatTok{4.6}\NormalTok{, }\FloatTok{4.7}\NormalTok{, }\FloatTok{5.0}\NormalTok{, }\FloatTok{5.5}\NormalTok{, }\FloatTok{7.2}\NormalTok{, }\FloatTok{7.8}\NormalTok{, }\FloatTok{8.6}\NormalTok{, }\FloatTok{9.4}\NormalTok{, }\FloatTok{10.0}\NormalTok{,  }\FloatTok{10.7}\NormalTok{, }\FloatTok{11.0}\NormalTok{, }
  \FloatTok{9.9}\NormalTok{, }\FloatTok{9.0}\NormalTok{, }\FloatTok{9.1}\NormalTok{, }\FloatTok{9.3}\NormalTok{, }\FloatTok{9.7}\NormalTok{, }\FloatTok{11.7}\NormalTok{, }
  \FloatTok{12.3}\NormalTok{, }\FloatTok{12.5}\NormalTok{, }\FloatTok{13.0}\NormalTok{, }\FloatTok{13.91}\NormalTok{, }\FloatTok{14.9}\NormalTok{, }\DecValTok{16}\NormalTok{, }\FloatTok{16.4}\NormalTok{, }\FloatTok{16.8}\NormalTok{)}


\KeywordTok{plot}\NormalTok{(x,y,}\DataTypeTok{type=}\StringTok{"l"}\NormalTok{,}\DataTypeTok{asp=}\DecValTok{1}\NormalTok{,}\DataTypeTok{main=}\StringTok{"Mano"}\NormalTok{)}

\KeywordTok{puntos}\NormalTok{(x,y)}
\end{Highlighting}
\end{Shaded}

\includegraphics{Documento_notebook_files/figure-latex/unnamed-chunk-8-1.pdf}

\begin{verbatim}
## desde :  1   punto medio :  2    hasta :  3  puntos intersectantes :  2  error segmento :  0 
## desde :  3   punto medio :  9    hasta :  11     puntos intersectantes :  0  error segmento :  3.013546 
## desde :  11  punto medio :  11   hasta :  12     puntos intersectantes :  1  error segmento :  0 
## desde :  12  punto medio :  13   hasta :  15     puntos intersectantes :  1  error segmento :  1.052632 
## desde :  15  punto medio :  18   hasta :  19     puntos intersectantes :  0  error segmento :  3.566667 
## desde :  19  punto medio :  20   hasta :  24     puntos intersectantes :  0  error segmento :  3.904167 
## desde :  24  punto medio :  24   hasta :  25     puntos intersectantes :  1  error segmento :  0 
## desde :  25  punto medio :  26   hasta :  28     puntos intersectantes :  1  error segmento :  2.9 
## desde :  28  punto medio :  32   hasta :  33     puntos intersectantes :  0  error segmento :  1.969444 
## desde :  33  punto medio :  36   hasta :  38     puntos intersectantes :  1  error segmento :  1.5 
## desde :  38  punto medio :  38   hasta :  39     puntos intersectantes :  1  error segmento :  0 
## desde :  39  punto medio :  42   hasta :  44     puntos intersectantes :  1  error segmento :  0.9388889 
## desde :  44  punto medio :  45   hasta :  49     puntos intersectantes :  1  error segmento :  3.208333 
## desde :  49  punto medio :  51   hasta :  52     puntos intersectantes :  0  error segmento :  0.9006944 
## desde :  52  punto medio :  53   hasta :  56     puntos intersectantes :  0  error segmento :  1.118631 
## desde :  56  punto medio :  62   hasta :  64     puntos intersectantes :  1  error segmento :  5.243333 
## desde :  64  punto medio :  64   hasta :  65     puntos intersectantes :  0  error segmento :  0.4 
## 
## 
## cantidad :  31   indice de Jaccard :  0.1692308 
## puntos seleccionados :  1 2 3 9 11 12 13 15 18 19 20 24 25 26 28 32 33 36 38 39 42 44 45 49 51 52 53 56 62 64 65
\end{verbatim}

-¿Su método es robusto, en el sentido que si se tienen más puntos la
exactitud no disminuye? Mientras los puntos que se presentan no sean
considerados como ruido o datos inexactos, es decir sean puntos
efectivamente que hacen parte de la mano, la exactitud del algoritmo es
capaz de al menos mantenerse e inclusive, debido a los segmentos más
cortos, llegará a ser más preciso.

-Si la información adicional o suponga tiene la información de otra mano
con más cifras significativas ¿como se comporta su algoritmo ? ¿la
exactitud decae? Al haber mayor cantidad de cifras significativas,
debido a que la aproximación hecha por el método utilizado a través de
spline, ésta información de menor cantidad de cifras al ser comparada
con valores de gran precisión podría no ser tan exacta, luego es posible
que exista mayor error entre los puntos dados y los graficados con la
aproximación del método.

\subsection{Bibliografía:}\label{bibliografia}

Para consultar el funcionamiento de spline se utilizó uno de los libros
proporcionados por la profesora del curso, el cual fue: Introducción a
los métodos numéricos por Walter Mora F.


\end{document}
